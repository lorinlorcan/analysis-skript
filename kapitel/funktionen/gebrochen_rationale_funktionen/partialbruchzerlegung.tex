\subsubsection{Partialbruchzerlegung}

\[
    y = \frac{5x + 11}{x^2 + 3x - 10}    
\]

\paragraph{I.\;Prüfen: Wirklich echt gebrochen?}

\textit{Ja}

\paragraph{II.\;Nullstellen des Nenners}

\begin{gather*}
    x^2 + 3x - 10 = 0 \\
    x_{1,2} = -\frac{3}{2} \pm \sqrt{\frac{9}{4}+ \frac{40}{4}} = -\frac{3}{2} \pm \frac{7}{2} \\
    x_1 = 2 \quad x_2 = -5
\end{gather*}

\[
    \Rightarrow \frac{5x + 11}{(x-2)(x + 5)}
\]

\paragraph{III.\;Aufstellen der Partialbrüche}

\begin{align*}
    \frac{5x + 11}{(x-2)(x + 5)} &= \frac{A}{x - 2} + \frac{B}{x + 5} &&\mid \cdot ((x-2)(x+5)) \\
    5x + 11 &= A(x + 5) + B(x - 2)
\end{align*}

\paragraph{IV.\;Konstanten bestimmen} \leavevmode \\
Durch Einsetzen von Werten (Optimalerweise den Nullstellen):

\begin{align*}
    5x + 11 &= A(x + 5) + B(x - 2) \\
    \\
    x = 2:\\
    21 &= 7A  \Leftrightarrow A = \frac{21}{7} = 3 \\
    x = -5:\\
    -14 &= -7B \Leftrightarrow B = 2
\end{align*}

\paragraph{V.\;Einsetzen}

\[
    y = \frac{5x + 11}{x^2 + 3x - 10} = \frac{3}{x-2} + \frac{2}{x+5}  
\]

\paragraph{Anmerkung}

Die Partialbruchzerlegung ist ein Lösungsverfahren für Integrale, ist normalerweise nicht von Vorteil für
die Kurvendiskussion:

\[
    \int \frac{5x + 11}{x^2 + 3x - 10} \diff x = \int \frac{3}{x-2} \diff x + \int \frac{2}{x+5} \diff x 
\]

\paragraph{Fallunterscheidung}

\subparagraph{1. Fall}

\(N(x)\) hat nur einfache reelle Nullstellen, \(x_i\) sei eine dieser
Nullstellen.

Dann gehört zu \(x_i\) ein Partialbruch der Form:

\[
    \frac{A_i}{x - x_i}    
\]

\subparagraph{2. Fall}

\(N(x)\) hat an der Stelle \(x_i\) eine \(k\)-fache Nullstelle.

Dann gehören zu \(x_i\) \(k\) Partialbrüche der Form:

\[
    \frac{A_{i1}}{x - x_i}  + \frac{A_{i2}}{{(x - x_i)}^2} + \frac{A_{i3}}{{(x - x_i)}^3} + \cdots + \frac{A_{ik}}{{(x - x_i)}^k}   
\]
