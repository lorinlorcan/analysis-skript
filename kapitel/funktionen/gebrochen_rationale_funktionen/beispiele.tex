\subsubsection{Beispiele}

\paragraph{1. Beispiel (einfachste gebrochene Funktion) (siehe Abb.~\ref{fig:gebrochenefunktionen_beispiel1})}

\[
    y = f(x) = \frac{1}{x}  
\]

punktsymmetrisch

\begin{figure}[H]
    \centering
    \begin{tikzpicture}
        \begin{axis}[
            default,
            xmin=-4.2, xmax=4.2,
            ymin=-4.2, ymax=4.2,
            width=6cm
            ]
            \draw [orange, smooth, samples=100, domain=-4:-1/100] plot (\x, {1 / \x});
            \draw [orange, smooth, samples=100, domain=1/100:4] plot (\x, {1 / \x});
        \end{axis}
    \end{tikzpicture}
    \caption{\(f(x) = \frac{1}{x}\)}\label{fig:gebrochenefunktionen_beispiel1}
\end{figure}

\paragraph{2. Beispiel  (siehe Abb.~\ref{fig:gebrochenefunktionen_beispiel2})}

\[
    y = f(x) = \frac{1}{x^2}  
\]

achsensymmetrisch

\begin{figure}[H]
    \centering
    \begin{tikzpicture}
        \begin{axis}[
            default,
            xmin=-4.2, xmax=4.2,
            ymin=-4.2, ymax=4.2,
            width=6cm
            ]
            \draw [orange, smooth, samples=100, domain=-4:-1/10] plot (\x, {1 /
            (pow(\x, 2))});
            \draw [orange, smooth, samples=100, domain=1/10:4] plot (\x, {1 /
            (pow(\x, 2))});
        \end{axis}
    \end{tikzpicture}
    \caption{\(f(x) = \frac{1}{x^2}\)}\label{fig:gebrochenefunktionen_beispiel2}
\end{figure}

\paragraph{2. Beispiel verschoben}

\[
    y = f(x) = \frac{1}{{(x-1)}^2}  
\]

\begin{figure}[H]
    \centering
    \begin{tikzpicture}
        \begin{axis}[
            default,
            xmin=-4.2, xmax=4.2,
            ymin=-4.2, ymax=4.2,
            width=6cm
            ]
            \draw [orange, smooth, samples=100, domain=-4:9/10] plot (\x, {1 /
            (pow((\x - 1), 2))});
            \draw [orange, smooth, samples=100, domain=11/10:4] plot (\x, {1 /
            (pow((\x - 1), 2))});
            \draw[dashed, gray] (1, -4) -- (1, 4);
        \end{axis}
    \end{tikzpicture}
    \caption{\(f(x) = \frac{1}{{(x-1)}^2}\)}\label{fig:gebrochenefunktionen_beispiel3}
\end{figure}
