\subsubsection{Beispiele}

\paragraph{1. Beispiel}

\[
	f(x) = 9x^2-x^4
\]

\begin{figure}[H]
	\centering
	\begin{tikzpicture}
		\begin{axis}[
				defaultpure,
				xmin=-4.2, xmax=4.2,
				ymin=-2.2, ymax=22.2,
				width=6cm
			]
			\draw [orange, smooth, samples=100, domain=-4:4] plot (\x, {9 * pow(\x, 2) -
					pow(\x, 4)});
		\end{axis}
	\end{tikzpicture}
	\caption{\(f(x) = 9x^2-x^4\)}\label{fig:nullstellen_grad4}
\end{figure}


\begin{align*}
	D_f = \{ x \mid x \in \mathbb{R} \}       \\
	\text{Symmetrie:} f(x) \text{ ist gerade} \\
	-x^4 \rightarrow \text{geht ins negative} \\
	\lim_{x\rightarrow\pm\infty} = - \infty
\end{align*}

Nullstellen:

\[
	f(x) = x^2(9-x^2) = x^2 (3^2 - x^2) = \underbrace{x^2}_{0\ \text{(doppelt)}} (3 - \underbrace{x}_{3}) (3 + \underbrace{x}_{-3})
\]

\paragraph{2. Beispiel}

\[
	f(x) = (x^2 - 1)(x^2 - 4) x
\]

\begin{figure}[H]
	\centering
	\begin{tikzpicture}
		\begin{axis}[
				default,
				xmin=-4.2, xmax=4.2,
				ymin=-4.2, ymax=4.2,
				width=6cm
			]
			\draw [orange, smooth, samples=100, domain=-2.5:2.5] plot (\x, {(pow(\x,
					2) - 1) * (pow(\x, 2) - 4) * \x});
		\end{axis}
	\end{tikzpicture}
	\caption{\(f(x) = (x^2 - 1)(x^2 - 4)x\)}\label{fig:nullstellen_grad5}
\end{figure}

\begin{align*}
	f(x) & = \underbrace{(x^2-1)}_{x = \pm 1}
	\underbrace{(x^2-4)}_{x = \pm 2}
	\underbrace{x}_{x = 0}                    \\
	     & = (x^4 - x^2 - 4x^2 + 4)x          \\
	     & = x^5 - x^3 - 4x^3 +4x             \\
	     & = x^5 - 5x^3 + 4x
\end{align*}

\begin{align*}
	\limtoinfty{x} f(x)    & = \infty  \\
	\limtomininfty{x} f(x) & = -\infty
\end{align*}

\begin{itemize}
	\item ungerade Exponenten: punktsymetrisch zum Ursprung
	\item Höchster Exponent:5 \\
	      \textrightarrow\ Funktion 5. Grades \\
	      \textrightarrow\ 5 Nullstellen
\end{itemize}
