\subsection{Logarithmusfunktionen}

Die Lösung \( x \) der Gleichung \( a^x = b \) nennt man
den Logarithmus von \( b \) zur Basis \( a \):

\[
    x = \log_a(b)
\]

Der Logarithmus von \( b \) zur Basis \( a \) ist jene reelle
Zahl \( x \) mit der man \( a \) potenzieren muss, um \( b \)
zu erhalten:

\[
    a^{\log_a(b)} = b    
\]

Die Logarithmusfunktion ist damit die Umkehrfunktion zur
Exponentialfunktion.


\begin{figure}[H]
    \centering
    \begin{tikzpicture}
        \begin{axis}[
                default,
                xmin=-2.2, xmax=2.2,
                ymin=-1.2, ymax=3.2,
                width=8cm
            ]
            \addplot[orange, smooth, samples=100, domain=0:2.2] {ln(x)};
            \addplot[cyan, smooth, samples=100, domain=0:2.2] {-ln(x)};
        \end{axis}
    \end{tikzpicture}
    \caption{Logarithmus steigend (\( \ln(x) \)) und fallend (\( -\ln{x} \))}
\end{figure}


\paragraph{Basis:}
\begin{equation*}
    \begin{array}{lll}
        a = e &\Rightarrow \log_e(x) &= \ln(x) \\
        a = 2 &\Rightarrow \log_2(x) &= \ld(x) = \lb(x) \\
        a = 10 &\Rightarrow \log_{10}(x) &= \lg(x)
    \end{array}
\end{equation*}

\subsubsection{Rechenregeln}

\begin{align*}
    \ln(a \cdot b) &= \ln(e^{\alpha} \cdot e^{\beta}) \\
    &= \ln(e^{(\alpha + \beta)}) \\
    &= \alpha + \beta \\
    &= \ln(e^{\alpha}) + \ln(e^{\beta}) \\
    &= \ln(a) + \ln (b) \\
    \\
    \ln(a^3) &= \ln(a^2 \cdot a) \\
    &= \ln(a^2) + \ln(a) \\
    &= \ln(a) + \ln(a) + \ln(a) \\
    &= 3 \cdot \ln(a) \\
    \\
    \ln(a^n) &= n \cdot \ln(a) \\
    \\
    \ln(\frac{a}{b}) &= \ln(a) + \ln(b^{-1}) \\
    &= \ln(a) - \ln(b)
\end{align*}

\subsubsection{Basiswechsel}
\[
    \log_a(x) \rightarrow \log_b(x)    
\]

\begin{align*}
    \log_a(x) = \log_a\left(b^{\log_b(x)}\right)
    &= \log_b(x) \cdot \log_a(b) \\
    \Rightarrow \log_a(x) &= \log_b(x) \cdot \frac{1}{\log_b(a)} \\
    \Rightarrow \log_a(x) &= \frac{\log_b(x)}{\log_b(a)} \\
    \text{wenn \(x = a\):} \\
    \log_a(a) &= \log_b(a) \cdot \log_a(b) = 1 \\
    \Rightarrow \log_a(b) &= \frac{1}{\log_b(a)} \\
    \\
    \text{Beispiel:} \\
    \log_3(5) &= \frac{\ln(5)}{\ln(3)}
\end{align*}

\paragraph{Beispiele}

\begin{align*}
    & \log(2) - \log(a) + \frac{1}{2} \log(b) \\
    &= \log\left(\frac{2}{a}\right) + \log\left(b^{\frac{1}{2}}\right) \\
    &= \log\left(\frac{2}{a}\right) + \log\left(\sqrt{b}\right) \\
    &= \log\left(\frac{2 \sqrt{b}}{a}\right)
\end{align*}

\begin{align*}
    \log(1-x) + \log(1+x) - 2 \log(x) \\
    &= \log((1-x)(1+x)) - \log\left(x^2\right) \\
    &= \log(1-x^2) - \log\left(x^2\right) \\
    &= \log\left(\frac{1-x^2}{x^2}\right) = \log\left(\frac{1}{x^2}-1\right)
\end{align*}

\begin{align*}
    10^x &= 2 \\
    \lg\left(10^x\right) &= \lg(2) \\
    x \lg(10) &= \lg(2) \\
    x \cdot 1 &= \lg(2) \\
    x &= \lg(2)
\end{align*}
