\subsection{Stetigkeit und Grenzwert von Funktionen}

\[
    \lim_{x \rightarrow x_0} f(x) = g
\]

\( f(x) \) hat an der Stelle \( x_0 \) den Grenzwert \( g \),
wenn es zu einem beliebig kleinen \( \epsilon > 0 \) eine Zahl \( \delta(\epsilon) \)
gibt, so dass \( \mid f(x) - g \mid < \epsilon \) ist,
\( \forall x \) für die \( \mid x - x_0 \mid < \delta \) gibt. 

\begin{figure}[H]
    \centering
    \begin{tikzpicture}
        \begin{axis}[
                default,
                xmin=-2.5, xmax=3.5,
                ymin=-0.5, ymax=3.5,
                width=8cm,
                xticklabels={},
                yticklabels={}
            ]
            \addplot[orange, smooth, samples=100, domain=-4:4] {1.5^x};
            
            \draw [dotted, thick] (1, -0.2) -- (1, 1.5);
            \draw [dotted, thick] (2, -0.2) -- (2, 2.25);
            \draw [dotted, thick] (3, -0.2) -- (3, 3.375);

            \draw [dotted, thick] (-0.2, 1.5) -- (1, 1.5);
            \draw [dotted, thick] (-0.2, 2.25) -- (2, 2.25);
            \draw [dotted, thick] (-0.2, 3.375) -- (3, 3.375);

            \node at (axis cs: 1, -0.3) {\( x_0 - \delta \)};
            \node at (axis cs: 2, -0.3) {\( x_0 \)};
            \node at (axis cs: 3, -0.3) {\( x_0 + \delta \)};

            \node[left] at (axis cs: -0.2, 1.5) {\( y_0 - \epsilon \)};
            \node[left] at (axis cs: -0.2, 2.25) {\( f(x_0) = y_0 \)};
            \node[left] at (axis cs: -0.2, 3.375) {\( y_0 + \epsilon \)};
        \end{axis}
    \end{tikzpicture}
    \caption{}
\end{figure}

\subsubsection{Definitionslücke}

\paragraph{Beispiel}

\[
	y = \frac{x^2 - 1}{x - 1}
\]

\begin{figure}[H]
	\centering
	\begin{tikzpicture}
		\begin{axis}[
				default,
				xmin=-0.5, xmax=3.5,
				ymin=-0.5, ymax=3.5,
				width=8cm
			]
			\addplot[orange, smooth, samples=100, domain=-4:4] {(x^2 - 1) / (x - 1)};
			\addplot[mark=o] coordinates {(1,2)};
			\node[right] at (axis cs: 1, 2) {Definitionslücke};
		\end{axis}
	\end{tikzpicture}
	\caption{\(\frac{x^2 - 1}{x - 1}\)}
\end{figure}

\[
	\lim_{x \rightarrow 1} \frac{x^2 - 1^2}{x - 1}
	= \lim_{x \rightarrow 1} \frac{(x - 1)(x + 1)}{x - 1}
	= \lim_{x \rightarrow 1} (x+1)
	= 2
\]

\textbf{rechtseitiger Grenzwert}

\[
	x > x_0: \lim_{x \rightarrow x_0} f(x) = 2 = g_{+}
\]

\textbf{rechtseitiger Grenzwert}

\[
	x < x_0: \lim_{x \rightarrow x_0} f(x) = 2 = g_{-}
\]


\[
	g_{+} = g_{-} \Rightarrow \text{\(f(x_0)\) ist nicht definiert}
\]

\subsubsection{Sprungstelle}

\begin{figure}[H]
	\centering
	\begin{tikzpicture}
		\begin{axis}[
				default,
				xmin=-0.5, xmax=3.5,
				ymin=-0.5, ymax=3.5,
				width=8cm,
				xticklabels={},
				yticklabels={}
			]
			\addplot[color1, smooth, samples=100, domain=-4:2] {2^x - 2};
			\addplot[color1, smooth, samples=100, domain=2:4] {-(x - 2)^2 + 3};

			\draw [dotted, thick] (2, -0.2) -- (2, 3.2);
			\node at (axis cs: 2, -0.3) {Sprungstelle};
		\end{axis}
	\end{tikzpicture}
	\caption{Funktion mit Sprungstelle}
\end{figure}

\[
	f(x) =
	\begin{cases}
		2^x - 2          & \quad x \leq 2 \\
		-{(x - 2)}^2 + 3 & \quad x > 2
	\end{cases}
\]

\[
	g_{+} \neq g_{-} \Rightarrow \text{Sprungstelle}
\]

\subsubsection{Polstelle}

\begin{figure}[H]
	\centering
	\begin{tikzpicture}
		\begin{axis}[
				default,
				xmin=-2.5, xmax=2.5,
				ymin=-2.5, ymax=2.5,
				width=8cm
			]
			\addplot[color1, smooth, samples=100, domain=-4:-0.1] {1/x};
			\addplot[color1, smooth, samples=100, domain=0.1:4] {1/x};
		\end{axis}
	\end{tikzpicture}
	\caption{\(y = \frac{1}{x}\)}
\end{figure}

\begin{align*}
	\begin{rcases}
		\lim_{x \rightarrow 0_+} f(x) &= \infty \\
		\lim_{x \rightarrow 0_-} f(x) &= -\infty
	\end{rcases}
	f(x), g\; \text{existieren nicht}
\end{align*}

