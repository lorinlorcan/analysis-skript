\subsubsection{Quadratische Funktion}

\[
	y = \underbrace{\underbrace{ax^2}_{\text{Potenzfunktion}} + bx + c}
	_{\text{Polynom / ganze Funktion}}
\]

\begin{figure}[H]
	\centering
	\begin{tikzpicture}
		\begin{axis}[
				defaultnonumbers,
				xmin=-2.2, xmax=6.2,
				ymin=-2.2, ymax=6.2,
				width=8.8cm
			]
			\draw [cyan, smooth, samples=100] plot (\x, {pow(\x, 2)});
			\draw [color1, smooth, samples=100] plot (\x, {pow(\x, 2) - 6 * \x + 11});
			\node at (axis cs: -2/3, 2) {\( b \)};
			\node at (axis cs: -2/3, 3) {\( y_0 \)};
			\node at (axis cs: 3, -2/3) {\( a \)};
			\node at (axis cs: 4, -2/3) {\( x_0 \)};
			\draw[-, gray] (0,2) -- (3,2);
			\draw[-, gray] (0,3) -- (4,3);
			\draw[-, gray] (3,0) -- (3,2);
			\draw[-, gray] (4,0) -- (4,3);
		\end{axis}
	\end{tikzpicture}
\end{figure}

\begin{align*}
	y         & = x^2               & \rightarrow y'   & = x'^2    \\
	x_0       & = a + x'_0          & \Rightarrow x'_0 & = x_0 - a \\
	y_0       & = b + y'_0          & \Rightarrow y'_0 & = y_0 - b \\
	\\
	y'        & = x'^2                                             \\
	(y_0 - b) & = {(x_0 -a)}^2                                     \\
	y_0       & = {(x_0 - a)}^2 + b                                \\
	y         & = {(x - a)}^2 + b
\end{align*}

\paragraph{Verschiebung auf der x-Achse}

\begin{description}[style=nextline]
	\item[negatives \(a\)] Verschiebung nach rechts
	\item[positives \(a\)] Verschiebung nach links
\end{description}

\paragraph{Herleitung der p-q-Formel}

\begin{align*}
	                &  & y = a x^2 + b x +c                      & = 0                                                                                             \\
	\\
	                &  & x^2 + px + q                            & = 0                                                                                             \\
	\Leftrightarrow &  & x^2 + px                                & = -q                                                       & \mid \text{Quadratische Ergänzung} \\
	\Leftrightarrow &  & x^2 + px + {\left(\frac{p}{2}\right)}^2 & = -q + {\left(\frac{p}{2}\right)}^2                                                             \\
	\Leftrightarrow &  & {\left( x+\frac{p}{2} \right)}^2        & = {\left(\frac{p}{2}\right)}^2 - q                                                              \\
	\Rightarrow     &  & x + \frac{p}{2}                         & = \pm \sqrt{{\left(\frac{p}{2}\right)}^2 - q}                                                   \\
	\Leftrightarrow &  & x_{1,2}                                 & = -\frac{p}{2} \pm \sqrt{{\left(\frac{p}{2}\right)}^2 - q}
\end{align*}

