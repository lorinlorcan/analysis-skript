\subsubsection{Sprungstelle}

\begin{figure}[H]
    \centering
    \begin{tikzpicture}
        \begin{axis}[
                default,
                xmin=-0.5, xmax=3.5,
                ymin=-0.5, ymax=3.5,
                width=8cm,
                xticklabels={},
                yticklabels={}
            ]
            \addplot[orange, smooth, samples=100, domain=-4:2] {2^x - 2};
            \addplot[orange, smooth, samples=100, domain=2:4] {-(x - 2)^2 + 3};
            
            \draw [dotted, thick] (2, -0.2) -- (2, 3.2);
            \node at (axis cs: 2, -0.3) {Sprungstelle};
        \end{axis}
    \end{tikzpicture}
    \caption{Funktion mit Sprungstelle}
\end{figure}

\[
    f(x) = 
    \begin{cases}
        2^x - 2 &\quad x \leq 2 \\
        -(x - 2)^2 + 3 &\quad x > 2
    \end{cases}
\]

\[
    g_{+} \neq g_{-} \Rightarrow \text{Sprungstelle}
\]
