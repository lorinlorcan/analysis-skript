\subsection{Exponentialfunktionen}

\paragraph{Allgemeine Form}

\[
    y = a^x \quad a \neq 1
\]

\begin{center}
    \( a > 0 \) damit wir in \( \mathbb{R} \) bleiben
    
    \( a \) wird Basis genannt, \( x \) Exponent

    \( a > 1 \) monoton steigend

    \( a < 1 \) monoton fallend
\end{center}

\begin{figure}[H]
    \centering
    \begin{tikzpicture}
        \begin{axis}[
                defaultnonumbers,
                xmin=-2.2, xmax=2.2,
                ymin=-1.2, ymax=3.2,
                width=8cm
            ]
            \addplot[orange, smooth, samples=100, domain=-4:4] {e^x};
            \addplot[cyan, smooth, samples=100, domain=-4:4] {e^-x};
            \node [right] at (axis cs: 1.2, 3) {\( a > 1 \)};
            \node [left] at (axis cs: -1.2, 3) {\( a < 1 \)};
        \end{axis}
    \end{tikzpicture}
    \caption{exponentiell steigend (\( e^x \)) und fallend( \( e^{-x} \))}
\end{figure}

\begin{figure}[H]
    \centering
    \begin{align*}
        y(t) &= {(1 + \omega)}^{t} \cdot y_0 \\
        \Delta y(t) &\sim y(t-1) \\
        \Delta y(t) &= \omega \cdot y(t-1) \quad \omega : \text{Wachstumsrate}
    \end{align*}
    \begin{tikzpicture}
        \begin{axis}[
                default,
                xmin=-1.2, xmax=6.2,
                ymin=-0.2, ymax=5.2,
                width=10cm,
                height=12cm,
                yticklabels={},
                xlabel={t},
            ]
            \addplot [orange, smooth, samples=100, domain=-4:4] {e^(0.5*x)};
            
            \node [left] at (axis cs: -0.2, 1.0000) {\( y(0) \)};
            \node [left] at (axis cs: -0.2, 1.6487) {\( y(1) \)};
            \node [left] at (axis cs: -0.2, 2.7183) {\( y(2) \)};
            \node [left] at (axis cs: -0.2, 4.4817) {\( y(3) \)};

            \draw [dotted, thick] (-0.2, 1.0000) -- (6.2, 1);
            \draw [dotted, thick] (-0.2, 1.6487) -- (6.2, 1.6487);
            \draw [dotted, thick] (-0.2, 2.7183) -- (6.2, 2.7183);
            \draw [dotted, thick] (-0.2, 4.4817) -- (6.2, 4.4817);

            \draw [decorate,decoration={brace,amplitude=4pt}] (1, 1.6487) -- (1, 1);
            \draw [decorate,decoration={brace,amplitude=4pt}] (2, 2.7183) -- (2, 1.6487);
            \draw [decorate,decoration={brace,amplitude=4pt}] (3, 4.4817) -- (3, 2.7183);
            
            \node [right] at (axis cs: 1.2, 1.3244) {\( \Delta y(1) = \omega \cdot y(0) \)};
            \node [right] at (axis cs: 2.2, 2.1835) {\( \Delta y(2) = \omega \cdot y(1) \)};
            \node [right] at (axis cs: 3.2, 3.6000) {\( \Delta y(3) = \omega \cdot y(2) \)};
        \end{axis}
    \end{tikzpicture}
    \[    
        \begin{alignedat}{5}
            \Delta y(3) &= y(3) - y(2) &= \omega \cdot y(2) \quad
            &\Rightarrow y(3) &= y(2) \cdot (1 + \omega) &= y(0) \cdot (1+\omega)^3 \\
            \Delta y(2) &= y(2) - y(1) &= \omega \cdot y(1) \quad
            &\Rightarrow y(2) &= y(1) \cdot (1 + \omega) &= y(0) \cdot (1+\omega)^2 \\
            \Delta y(1) &= y(1) - y(0) &= \omega \cdot y(0) \quad
            &\Rightarrow y(1) &= y(0) \cdot (1 + \omega) &= y(0) \cdot (1+\omega)            
        \end{alignedat}
    \]
    % \caption{}
\end{figure}

\begin{uebung}[1]
    \begin{equation*}
        \begin{array}{lrl}
            & 9^{x-1} &= 27 \\
            \Leftrightarrow& 9^{x-1} &= 3^3 \\
            \Leftrightarrow& {(3^2)}^{x-1} &= 3^3 \\
            \Leftrightarrow& 3^{2x-2} &= 3^3 \\
            \Rightarrow& 2x-2 &= 3 \\
            \Leftrightarrow& 2x &= 5 \\
            \Leftrightarrow& x &= \frac{5}{2}
        \end{array}
    \end{equation*}
\end{uebung}

\begin{uebung}[2]
    \begin{equation*}
        \begin{array}{lrl}
            & {\left(\frac{4}{9}\right)}^{x-2} &= \frac{8}{27} \\
            \Leftrightarrow& {\left(\frac{4}{9}\right)}^{x-2} &=
            {\left(\frac{2}{3}\right)}^{3} \\
            \Leftrightarrow& \left({\left(\frac{2}{3}\right)}^{2}\right)^{x-2} &=
            {\left(\frac{2}{3}\right)}^{3} \\
            \Leftrightarrow& {\left(\frac{2}{3}\right)}^{2x-4} &=
            {\left(\frac{2}{3}\right)}^{3} \\
            \Rightarrow& 2x-4 &= 3 \\
            \Leftrightarrow& 2x &= 7 \\
            \Leftrightarrow& x &= \frac{7}{2}
        \end{array}
    \end{equation*}
\end{uebung}

\begin{uebung}[3]
    \begin{equation*}
        \begin{array}{lrlr}
            & 8 \cdot 9^{x-3} + 4^{x-3} &= 3^{2x-4} & \\
            \Leftrightarrow& 8 \cdot 3^{2x-9} + 4^{x-3} &= 3^{2x-4} & \\
            \Leftrightarrow& 2^3 \cdot 3^{2x-9} + 2^{2x-6} &= 3^{2x-4} &\mid -(2^3 \cdot 3^{2x-9}) \\
            \Leftrightarrow& 2^{2x-6} &= 3^{2x-4} - 2^3 \cdot 3^{2x-9} & \\
            \Leftrightarrow& 2^{2x-6} &= 3^{2} \cdot 3^{2x-6} - 2^{3} \cdot 3^{2x-9} \\
            \Leftrightarrow& 2^{2x-6} &= (3^2 - 2^3)
            % TODO: siehe Zettel - ergänzen 
        \end{array}
    \end{equation*}
\end{uebung}
