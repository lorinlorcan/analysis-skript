\subsubsection{Überblick}

\begin{figure}[H]
	\centering
	\begin{tikzpicture}
		\tikzstyle{polstelle}=[dotted, thick]
		\begin{axis}[
				default,
				xmin=-0.2, xmax=8.2,
				ymin=-2.2, ymax=2.2,
				width=12cm
			]
			\draw [mkgreen, smooth, samples=100, domain=-1:3*pi] plot (\x, {sin(\x r)});
			\draw [mkblue, smooth, samples=100, domain=-1:3*pi] plot (\x, {cos(\x r)});
			\draw [mkyellow, smooth, samples=100, domain=-1:5/12*pi] plot (\x, {tan(\x r)});
			\draw [mkyellow, smooth, samples=100, domain=7/12*pi:17/12*pi] plot (\x, {tan(\x r)});
			\draw [mkyellow, smooth, samples=100, domain=19/12*pi:29/12*pi] plot (\x, {tan(\x r)});
			\draw [mkred, smooth, samples=100, domain=1/12*pi:11/12*pi] plot
			(\x, {cos(\x r) / sin(\x r)});
			\draw [mkred, smooth, samples=100, domain=13/12*pi:23/12*pi] plot
			(\x, {cos(\x r) / sin(\x r)});
			\draw [mkred, smooth, samples=100, domain=25/12*pi:35/12*pi] plot
			(\x, {cos(\x r) / sin(\x r)});
			\draw[mkred, polstelle] (pi,-2.5) -- (pi,2.5);
			\draw[mkred, polstelle] (2*pi,-2.5) -- (2*pi,2.5);
			\draw[mkred, polstelle] (3*pi,-2.5) -- (3*pi,2.5);
			\draw[mkyellow, polstelle] (1/2*pi,-2.5) -- (1/2*pi,2.5);
			\draw[mkyellow, polstelle] (3/2*pi,-2.5) -- (3/2*pi,2.5);
			\draw[mkyellow, polstelle] (5/2*pi,-2.5) -- (5/2*pi,2.5);

			\draw[] (pi,-0.1) -- (pi,0.1);
			\node[] at (axis cs: pi, 0.3) {\(\pi\)};
			\draw[] (2*pi,-0.1) -- (2*pi,0.1);
			\node[] at (axis cs: 2*pi, 0.3) {\(2\pi\)};
		\end{axis}
	\end{tikzpicture}
	\caption{Die trigonometrischen Funktionen}
	\textbf{\color{mkgreen}---} \(\sin(x)\)
	\textbf{\color{mkblue}---} \(\cos(x)\)
	\textbf{\color{mkyellow}---} \(\tan(x)\)
	\textbf{\color{mkred}---} \(\cot(x)\)
\end{figure}



\begin{center}
	\begin{tabular}{ c c c c  }
		Funktion      & Definitionsbereich               & Wertebereich                     & Symmetrie \\
		\toprule
		\( \sin(x) \) & \( -\infty \leq x \leq \infty \) & \( -1 \leq y \leq 1
		\)            & Punktsymmetrie                                                                  \\
		\( \cos(x) \) & \( -\infty \leq x \leq \infty \) & \( -1 \leq y \leq 1
		\)            & Achsensymmetrie                                                                 \\
		\( \tan(x) \) & \( x \neq \frac{k}{2} \pi \)     & \( -\infty \leq y \leq
		\infty \)     & Punktsymmetrie                                                                  \\
		\( \cot(x) \) & \( x \neq k \pi \)               & \( -\infty \leq y \leq \infty \) &
		Punktsymmetrie                                                                                  \\
	\end{tabular}
\end{center}
