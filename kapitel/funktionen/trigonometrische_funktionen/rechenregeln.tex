\subsubsection{Rechenregeln}

\paragraph{Additionstheorem für Sinus}

\[
    \sin(\alpha + \beta) = \sin \alpha \cdot \cos \beta + \sin \beta \cdot \cos \alpha 
\]

\begin{figure}[H]
    \centering
    \begin{tikzpicture}[font=\sffamily]
        \begin{axis}[
                defaultnonumbers,
                xmin=-0.2, xmax=1.2,
                ymin=-0.2, ymax=1.2,
                width=8cm,
                grid=none
            ]
            \draw (1, 0) arc (0:360:1);
            
            % Strecke ME
            \draw (0,0) -- (1.2, 0.8);
            % Strecke MP
            \draw (0,0) -- (0.7, 1.1);
            
            % Winkel Alpha
            \draw[->] (0.4,0) arc (0:33.69:0.4);
            \node at (axis cs: 0.3, 0.08) {\textalpha};
            % Winkel Beta
            \draw[->] (0.33,0.22) arc (33.69:57.53:0.4);
            \node at (axis cs: 0.22, 0.22) {\textbeta};
            
            % Punktbeschriftung
            \node at (axis cs: 1.05, -0.1) {1};
            \node at (axis cs: -0.05, -0.05) {M};
            \node at (axis cs: 0.83, 0.65) {E};
            \node at (axis cs: 0.53, 0.9) {P};
            \node at (axis cs: 0.53, -0.05) {A};
            \node at (axis cs: 0.76, -0.05) {F};
            \node at (axis cs: 0.8, 0.45) {B};
            \node at (axis cs: 0.5, 0.5) {T};

            % Strecke AP
            \draw[mkred, thick] (0.54, 0) -- (0.54, 0.84);
            % Strecke PE
            \draw[mkblue, thick] (0.54, 0.84) -- (0.76, 0.5);
            % Rechter Winkel an B
            \draw[mkblue, thick] (0.71, 0.58) arc (123.69: 213: 0.085);
            \node[mkblue, thick] at (axis cs: 0.71, 0.51) {\( \cdot \)};
            % Strecke BF
            \draw[mkyellow, thick] (0.76, 0.5) -- (0.76, 0);
            % Strecke TB
            \draw[mkgreen, thick] (0.54, 0.5) -- (0.76, 0.5);
            % Markierung Strahlensatz 
            \draw[gray] (0.54, 0.42) -- (0.59, 0.39);
            \draw[gray] (0.49, 0.33) -- (0.54, 0.3);
            
            % Winkel Alpha an P
            \draw[->] (0.54, 0.7) arc (270: 303.69: 0.14);
            \node at (axis cs: 0.7, 0.8) {\textalpha};
            % Beschriftungslinie für Alpha
            \draw[thin] (0.56, 0.75) -- (0.67, 0.78);
        \end{axis}
    \end{tikzpicture}
    \caption{Herleitung des Additionstheorems}
\end{figure}


\[
    \underbrace{\sin(\alpha + \beta)}_{\overline{PA}} =
    \underbrace{\sin \alpha \cdot \cos \beta}_{\overline{AT}} +
    \underbrace{\sin \beta \cdot \cos \alpha}_{\overline{PT}} 
\]

\begin{align*}
    \sin \alpha &= \frac{\overline{BF}}{\overline{MB}}
    &\sin \beta &= \frac{\overline{PB}}{\overline{MP}} \\
    \cos \alpha &= \frac{\overline{PT}}{\overline{PB}}
    &\cos \beta &= \frac{\overline{MB}}{\overline{MP}}
\end{align*}

\begin{align*}
    \sin \alpha = \frac{\overline{BF}}{\overline{MB}}
    \Leftrightarrow \overline{BF} = \sin \alpha \cdot \overline{MB} 
    = \sin \alpha \cdot \cos \beta \\
    \cos \alpha = \frac{\overline{PT}}{\overline{PB}} 
    \Leftrightarrow \overline{PT} = \cos \alpha \cdot \overline{PB}
    = \cos \alpha \cdot \sin \beta
\end{align*}

\paragraph{Beispiele}

\begin{align*}
    \sin(2 \alpha)
    &= \sin\alpha \cdot \cos\alpha + \sin\alpha \cdot \cos\alpha \\
    &= 2 \cdot \sin\alpha \cdot \cos\alpha \\
    \\
    \cos(\alpha + \beta)
    &= \sin\left(\alpha + \beta + \frac{\pi}{2}\right) \\
    &= \sin\left(\alpha + \left(\beta + \frac{\pi}{2}\right)\right) \\
    &= \sin\alpha \cdot \cos\left( \beta + \frac{\pi}{2} \right)
     + \sin\left( \beta + \frac{\pi}{2} \right) \cdot \cos\alpha \\
    &= \sin\alpha \cdot (-\sin\beta) + \cos\beta \cdot \cos\alpha \\
    &= \cos\alpha \cdot \cos\beta - \sin\alpha \cdot \sin\beta
\end{align*}
