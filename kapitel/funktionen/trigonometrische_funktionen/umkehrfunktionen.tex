\subsubsection{Umkehrfunktionen}

\begin{figure}[H]
	\centering
	\begin{tikzpicture}[font=\sffamily]
		\begin{axis}[
				defaultnonumbers,
				xmin=-0.2, xmax=1.2,
				ymin=-0.2, ymax=1.2,
				width=8cm,
				xlabel={}, ylabel={}
			]
			\draw (1, 0) arc (0:360:1);
			\draw (0, 0) -- (1.2, 1);
			\node at (axis cs: -1.1, 0.1) {1};
			\node at (axis cs: -0.1, -0.1) {M};
			\node at (axis cs: 0.6, 0.64) {P};
			\node at (axis cs: 0.77, -0.1) {A};
			\node at (axis cs: 0.2, 0.08) {x};
			\draw (0.3,0) arc (0:39.81:0.3);

			\draw[color2, thick] (1, 0) arc (0:39.81:1);
			\node[color2, rotate=90] at (axis cs: 1.1, 0.25) {arcsin(x)};
			\draw[mkgreen, thick] (0.77,0) -- (0.77, 0.64);
			\node[mkgreen, left] at (axis cs: 0.8, 0.25) {sin(x)};
		\end{axis}
	\end{tikzpicture}
	\caption{Herleitung von \( \arcsin \) am Einheitskreis}
\end{figure}

\begin{figure}[H]
	\centering
	\begin{tikzpicture}[font=\sffamily]
		\begin{axis}[
				defaultnonumbers,
				xmin=-0.2, xmax=1.2,
				ymin=-0.2, ymax=1.2,
				width=8cm,
				xlabel={}, ylabel={}
			]
			\addplot [domain=-1:1, samples=100] ({cos(x)},{sin(x)});

			\draw (0,0) -- (1.2, 1);
			\node at (axis cs: -1.1, 0.1) {1};
			\node at (axis cs: -0.1, -0.1) {M};
			\node at (axis cs: 0.6, 0.64) {P};
			\node at (axis cs: 0.77, -0.1) {A};
			\node at (axis cs: 0.2, 0.08) {x};
			\draw (0.3,0) arc (0:39.81:0.3);

			\draw[mkred, thick] (1, 0) arc (0:39.81:1);
			\node[mkred, rotate=90] at (axis cs: 1.1, 0.25) {arccos(x)};
			\draw[mkgreen, thick] (0,0) -- (1, 0);
			\node[mkgreen] at (axis cs: 0.5, -0.1) {cos(x)};
		\end{axis}
	\end{tikzpicture}
	\caption{Herleitung von \( \arccos \) am Einheitskreis}
\end{figure}

\begin{gesetz}
	\begin{align*}
		\arcsin(x) + \arccos(x) = \frac{\pi}{2} \\
		\arctan(x) + \arccot(x) = \frac{\pi}{2}
	\end{align*}
\end{gesetz}

\begin{achtung}
	Auf dem Taschenrechner und auch in mancher Literatur findet sich die
	Schreibweise \( \arcsin = \sin^{-1} \), dass bedeutet jedoch nicht \(
	\frac{1}{\sin} \).
\end{achtung}

\begin{figure}[H]
	\centering
	\begin{tikzpicture}
		\tikzstyle{polstelle}=[dotted, thick]
		\begin{axis}[
				default,
				xmin=-2.2, xmax=4.2,
				ymin=-2.2, ymax=4.2,
				width=10cm,
				xtick distance=1,
				ytick distance=1,
				extra y tick style={grid=none},
				extra y ticks={pi, -pi/2, pi/2},
				extra y tick labels={\(\pi\),\(-\frac{\pi}{2}\),\(\frac{\pi}{2}\)}
			]
			\draw[mkyellow, very thin, smooth, samples=100, domain=-4.2:4.2]
			plot (\x, {sin(\x r)});

			\draw[mkred, very thin, smooth, samples=100, domain=-4.2:4.2]
			plot (\x, {cos(\x r)});

			\draw[color2, very thick, smooth, samples=100, domain=-1:1]
			plot (\x, {rad(asin(\x))});

			\draw[mkgreen, very thick, smooth, samples=100, domain=-1:1]
			plot (\x, {rad(acos(\x))});

			\draw[mkgreen, polstelle] (-2.2, pi) -- (4.2, pi);
			\draw[color2, polstelle] (-2.2, pi/2) -- (4.2, pi/2);
			\draw[color2, polstelle] (-2.2, -pi/2) -- (4.2, -pi/2);
		\end{axis}
	\end{tikzpicture}
	\caption{Die Umkehrfunktionen zu Sinus und Cosinus}
	\textbf{\color{mkblue}---} \(\arcsin(x)\)
	\textbf{\color{mkgreen}---} \(\arccos(x)\)
	\textbf{\color{mkyellow}---} \(\sin(x)\)
	\textbf{\color{mkred}---} \(\cos(x)\)
\end{figure}

\begin{figure}[H]
	\centering
	\begin{tikzpicture}
		\tikzstyle{polstelle}=[dotted, thick]
		\tikzstyle{standardfunktion}=[very thin]
		\begin{axis}[
				default,
				xmin=-4.2, xmax=4.2,
				ymin=-4.2, ymax=4.2,
				width=10cm,
				extra x tick style={grid=none},
				extra x ticks={-pi, pi},
				extra x tick labels={\(-\pi\),\(\pi\)},
				extra y tick style={grid=none},
				extra y ticks={pi, -pi/2, pi/2},
				extra y tick labels={\(\pi\),\(-\frac{\pi}{2}\),\(\frac{\pi}{2}\)}
			]
			\draw[mkyellow, standardfunktion, smooth, samples=100, domain=-4.2:-1.6]
			plot (\x, {tan(\x r)});
			\draw[mkyellow, standardfunktion, smooth, samples=100, domain=-1.4:1.4]
			plot (\x, {tan(\x r)});
			\draw[mkyellow, standardfunktion, smooth, samples=100, domain=1.6:4.2]
			plot (\x, {tan(\x r)});

			% \draw[mkyellow, polstelle] (-1/2*pi, -4.2) -- (-1/2*pi, 4.2);
			% \draw[mkyellow, polstelle] (1/2*pi, -4.2) -- (1/2*pi, 4.2);

			\draw[mkblue, very thick, smooth, samples=100, domain=-4.2:4.2]
			plot (\x, {rad(atan(\x)});

			\draw[mkgreen, very thick, smooth, samples=100, domain=-4.2:4.2]
			plot (\x, {rad(90-atan(\x))});

			\draw [mkred, standardfunktion, smooth, samples=100, domain=-4.2:-3.2]
			plot (\x, {cos(\x r) / sin(\x r)});
			\draw [mkred, standardfunktion, smooth, samples=100, domain=-3:-0.1]
			plot (\x, {cos(\x r) / sin(\x r)});
			\draw [mkred, standardfunktion, smooth, samples=100, domain=0.1:3]
			plot (\x, {cos(\x r) / sin(\x r)});
			\draw [mkred, standardfunktion, smooth, samples=100, domain=3.2:4.2]
			plot (\x, {cos(\x r) / sin(\x r)});

			% \draw[mkred, polstelle] (-pi, -4.2) -- (-pi, 4.2);
			% \draw[mkred, polstelle] (pi, -4.2) -- (pi, 4.2);

			% Vertikale Polstellen
			\draw[mkgreen, polstelle] (-4.2, pi) -- (4.2, pi);
			\draw[mkblue, polstelle] (-4.2, pi/2) -- (4.2, pi/2);
			\draw[mkblue, polstelle] (-4.2, -pi/2) -- (4.2, -pi/2);
		\end{axis}
	\end{tikzpicture}
	\caption{Die Umkehrfunktionen zu Tangens und Cotangens}
	\textbf{\color{mkblue}---} \(\arctan(x)\)
	\textbf{\color{mkgreen}---} \(\arccot(x)\)
	\textbf{\color{mkyellow}---} \(\tan(x)\)
	\textbf{\color{mkred}---} \(\cot(x)\)
\end{figure}

\begin{uebung}
	Vereinfachen.

	\begin{question}
		\[
			\frac{\sin(x)}{\cos(x)}
		\]
	\end{question}

	\begin{solution}
		\begin{align*}
			\frac{\sin(x)}{\cos(x)}                              \\
			 & = \frac{\tan(x) \cdot \cos(x)}{\tan(x)} = \cos(x)
		\end{align*}
	\end{solution}

	\begin{question}
		\[
			\frac{\cos(x)}{\cot(x)}
		\]
	\end{question}

	\begin{solution}
		\begin{align*}
			\frac{\cos(x)}{\cot(x)}                              \\
		 	& = \frac{\cos(x) \cdot \sin(x)}{\cos(x)} = \sin(x)
		\end{align*}
	\end{solution}

	\begin{question}
		\[
			\sqrt{1+ \tan^2(x)} \cdot \cos(x)
		\]
	\end{question}

	\begin{solution}
		\begin{align*}
			\sqrt{1+ \tan^2(x)} \cdot \cos(x)                                         \\
			 & = \sqrt{1 + \frac{\sin^2(x)}{\cos^2(x)}} \cdot \sqrt{\cos^2(x)}        \\
			 & = \sqrt{\left(1 + \frac{\sin^2(x)}{\cos^2{x}} \right) \cdot \cos^2(x)} \\
			 & = \sqrt{\cos^2(x) + \sin^2(x)} = \sqrt{1} = 1
		\end{align*}
	\end{solution}

	\begin{question}
		\[
			\sqrt{1 + \tan^2(x)} \cdot \cos(x)
		\]
	\end{question}

	\begin{solution}
		\begin{align*}
			\sqrt{1 + \tan^2(x)} \cdot \cos(x)   \\
			 & = (1 + \tan(x)) \cdot \cos(x)     \\
			 & = \cos(x) + \tan(x) \cdot \cos(x) \\
			 & = \cos(x) + \sin(x)
		\end{align*}
	\end{solution}
\end{uebung}
