\subsubsection{Tangens \& Cotangens}

\begin{figure}[H]
    \centering
    \begin{tikzpicture}
        \tikzstyle{polstelle}=[dotted, thick]
        \begin{axis}[
            default,
            xmin=-1.2, xmax=7.2,
            ymin=-2.2, ymax=2.2,
            width=12cm
            ]
            \draw [mkyellow, smooth, samples=100, domain=-1.2:5/12*pi] plot (\x, {tan(\x r)});
            \draw [mkyellow, smooth, samples=100, domain=7/12*pi:17/12*pi] plot (\x, {tan(\x r)});
            \draw [mkyellow, smooth, samples=100, domain=19/12*pi:29/12*pi] plot (\x, {tan(\x r)});
            \draw [mkred, smooth, samples=100, domain=-11/12*pi:-1/12*pi] plot
            (\x, {cos(\x r) / sin(\x r)});
            \draw [mkred, smooth, samples=100, domain=1/12*pi:11/12*pi] plot
            (\x, {cos(\x r) / sin(\x r)});
            \draw [mkred, smooth, samples=100, domain=13/12*pi:23/12*pi] plot
            (\x, {cos(\x r) / sin(\x r)});
            \draw [mkred, smooth, samples=100, domain=25/12*pi:35/12*pi] plot
            (\x, {cos(\x r) / sin(\x r)});
            \draw[mkred, polstelle] (pi,-2.5) -- (pi,2.5);
            \draw[mkred, polstelle] (2*pi,-2.5) -- (2*pi,2.5);
            \draw[mkred, polstelle] (3*pi,-2.5) -- (3*pi,2.5);
            \draw[mkyellow, polstelle] (1/2*pi,-2.5) -- (1/2*pi,2.5);
            \draw[mkyellow, polstelle] (3/2*pi,-2.5) -- (3/2*pi,2.5);
            \draw[mkyellow, polstelle] (5/2*pi,-2.5) -- (5/2*pi,2.5);

            \draw[] (pi,-0.1) -- (pi,0.1);
            \node[] at (axis cs: pi, 0.3) {\(\pi\)};
            \draw[] (2*pi,-0.1) -- (2*pi,0.1);
            \node[] at (axis cs: 2*pi, 0.3) {\(2\pi\)};
        \end{axis}
    \end{tikzpicture}
    \caption{Tangens und Cotangens}
    \textbf{\color{mkyellow}---} \(\tan(x)\)
    \textbf{\color{mkred}---} \(\cot(x)\)
\end{figure}

\begin{align*}
    \tan(-x) &= -\tan(x) &&\Rightarrow \text{punktsymmetrisch} \\
    \cot(-x) &= -\cot(x) &&\Rightarrow \text{punktsymmetrisch} \\
    \\
    \tan(x + k \pi) &= \tan(x) &&k \in \mathbb{Z} \\
    \cot(x + k \pi) &= \cot(x) &&k \in \mathbb{Z} \\
\end{align*}

\begin{center}
    \(\tan(x)\) und \(\cot(x)\) sind periodisch mit \(\pi \).
\end{center}
