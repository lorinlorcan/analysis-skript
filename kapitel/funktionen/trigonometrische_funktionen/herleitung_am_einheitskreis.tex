\subsubsection{Herleitung am Einheitskreiss}

Der Einheitskreis wird beschrieben durch:

\[
	y^2+x^2 = 1
\]

oder auch

\[
	\sin^2(\alpha) + \cos^2(\alpha) = 1
\]

\begin{figure}[H]
	\centering
	\begin{tikzpicture}
		\begin{axis}[
				defaultnonumbers,
				xmin=-1.2, xmax=1.2,
				ymin=-1.2, ymax=1.2,
				width=8cm
			]
			\draw (1, 0) arc (0:360:1);
			\draw (0,0) -- (1.2, 1);
			\node at (axis cs: -1.1, 0.1) {1};
			\node at (axis cs: -0.1, -0.1) {M};
			\node at (axis cs: -0.1, 1.1) {D};
			\node at (axis cs: 1.1, -0.1) {B};
			\node at (axis cs: 0.9, 10/11) {C};
			\node at (axis cs: 0.6, 0.64) {P};
			\node at (axis cs: 0.77, -0.1) {A};
			\node at (axis cs: 1.1, 1.1) {E};
			\node at (axis cs: 0.2, 0.08) {\textalpha};
			\draw (0.3,0) arc (0:39.81:0.3);
			\draw[mkred, thick] (0,1) -- (1.2, 1);
			\node[mkred] at (axis cs: 0.5, 1.1) {cot(\textalpha)};
			\draw[mkgreen, thick] (0.77,0) -- (0.77, 0.64);
			\node[mkgreen, left] at (axis cs: 0.8, 0.25) {sin(\textalpha)};
			\draw[mkyellow, thick] (1,0) -- (1, 5/6);
			\node[mkyellow, rotate=90] at (axis cs: 1.1, 0.4) {tan(\textalpha)};
			\draw[mkblue, thick] (0,0) -- (0.77,0);
			\node[mkblue, below] at (axis cs: 0.4, 0) {cos(\textalpha)};
		\end{axis}
	\end{tikzpicture}
	\caption{Herleitung der trigonometrischen Funktionen am Einheitskreis}
\end{figure}

Die Funktionen:

\begin{itemize}
	\item {
	      \color{mkgreen}
	      \textbf{Sinus:}
	      }\\
	      \(\overline{PA} = \sin(\alpha)\)
	\item {
	      \color{mkblue}
	      \textbf{Cosinus\footnote{wird auch \enquote{Kosinus} geschrieben}:}
	      }\\
	      \(\overline{MA} = \cos(\alpha)\)
	\item {
	      \color{mkyellow}
	      \textbf{Tangens:}
	      }\\
	      \(\overline{BC} = \tan(\alpha)\)
	\item {
	      \color{mkred}
	      \textbf{Cotangens\footnote{wird auch \enquote{Kotangens} geschrieben}:}
	      }\\
	      \(\overline{DE} = \cot(\alpha)\)
\end{itemize}


\begin{align*}
	\frac{\sin(\alpha)}{\cos(\alpha)} & = \tan(\alpha) \\
	\frac{\cos(\alpha)}{\sin(\alpha)} & = \cot(\alpha)
\end{align*}
