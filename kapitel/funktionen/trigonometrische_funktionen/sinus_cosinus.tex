\subsubsection{Sinus \& Cosinus}

\begin{figure}[H]
	\centering
	\begin{tikzpicture}
		\begin{axis}[
				default,
				xmin=-1.2, xmax=7.2,
				ymin=-1.2, ymax=1.2,
				width=12cm
			]
			\draw [color1, smooth, samples=100, domain=-1.2:3*pi] plot (\x, {sin(\x r)});
			\draw [color2, smooth, samples=100, domain=-1.2:3*pi] plot (\x, {cos(\x r)});

			\draw[] (pi,-0.1) -- (pi,0.1);
			\node[] at (axis cs: pi, 0.3) {\(\pi\)};
			\draw[] (2*pi,-0.1) -- (2*pi,0.1);
			\node[] at (axis cs: 2*pi, 0.3) {\(2\pi\)};
		\end{axis}
	\end{tikzpicture}
	\caption{Sinus und Cosinus}
	\textbf{\color{color1}---} \(\sin(x)\)
	\textbf{\color{color2}---} \(\cos(x)\)
\end{figure}

\begin{align*}
	\sin(-x)         & = -\sin(x) &  & \Rightarrow \text{punktsymmetrisch}  \\
	\cos(-x)         & = \cos(x)  &  & \Rightarrow \text{achsensymmetrisch} \\
	\\
	\sin(x + 2k \pi) & = \sin(x)  &  & k \in \mathbb{Z}                     \\
	\cos(x + 2k \pi) & = \cos(x)  &  & k \in \mathbb{Z}                     \\
\end{align*}

\begin{center}
	\(\sin(x)\) und \(\cos(x)\) sind periodisch mit \(2\pi \).
\end{center}
