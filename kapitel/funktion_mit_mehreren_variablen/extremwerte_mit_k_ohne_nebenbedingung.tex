\subsection{Extremwerte mit k ohne Nebenbedingung}

\begin{align*}
	\frac{\partial f}{\partial x} & = 0 \\
	\frac{\partial f}{\partial y} & = 0 \\
\end{align*}

\paragraph{Hinreichende Bedingung}

% TODO: Korrigieren!

\begin{align*}
	f_{xx} f_{yy} - f_{{(xy)}^2} & > 0 \Rightarrow \text{Extremwert}  \\
	                             & < 0 \Rightarrow \text{Sattelpunkt}
\end{align*}

\begin{align*}
	\begin{rcases}
		f_{xx} (P) &< 0 \\
		\text{oder}\quad f_{yy} (P) &< 0
	\end{rcases}
	 & \text{Maximum} \\
	\begin{rcases}
		f_{xx} (P) &> 0 \\
		\text{oder}\quad f_{yy} (P) &> 0
	\end{rcases}
	 & \text{Minimum}
\end{align*}

\paragraph{Beispiel}

\subparagraph{1)}

\begin{align*}
	z   & = f(x, y) = 2x + y - \frac{x^2 + y^2}{4}               \\
	f_x & = 2 - \frac{x}{2} \overset{!}{=} 0 \Rightarrow x_0 = 4 \\
	f_z & = 1 - \frac{y}{2} \overset{!}{=} 0 \Rightarrow y_0 = 2 \\
	\\
	R   & = \{ (x, y) x^2 + y^2 - 9 = 0 \}                       \\
	    & 0 \leq x, y \leq 3
\end{align*}

\begin{align*}
	L(x, y, \lambda) & = f(x, y) + \lambda \cdot R(x, y)                       \\
	                 & = 2x + y - \frac{x^2 + y^2}{4} + \lambda(x^2 + y^2 - 9) \\
	\begin{rcases}
		\frac{\partial L}{\partial x} &\overset{!}{=} 0 \\
		\frac{\partial L}{\partial y} &\overset{!}{=} 0 \\
		\frac{\partial L}{\partial \lambda} &\overset{!}{=} 0
	\end{rcases}
	                 & x, y, \lambda                                           \\
	x                & = \frac{2}{\frac{1}{x} - 2 \lambda}                     \\
	y                & = \frac{2}{\frac{1}{2} + 2 \lambda}
\end{align*}

\subparagraph{2)}

\begin{align*}
	U(x, y) = 2 \cdot x \cdot y                                \\
	p_x = 3, p_y = 2 \qquad & C = 60 = p_x x + p_y y = 3x + 2y \\
	                        & \Rightarrow R = 3x + 2y - 60
\end{align*}
\begin{align*}
	L(x, y, \lambda) = 2xy + \lambda (3x + 2y - 60)                                                                                            \\
	L_x = 2 y + 3 \lambda      & \overset{!}{=} 0 &  & \Rightarrow y = - \frac{3}{2} \lambda &  & \Rightarrow y = 15                           \\
	L_y = 2 x + 3 \lambda      & \overset{!}{=} 0 &  & \Rightarrow x = - \lambda             &  & \Rightarrow x = 10                           \\
	L_\lambda = 3 x + 2 y - 60 & \overset{!}{=} 0 &  &                                       &  & \Rightarrow - 3 \lambda - 3 \lambda - 60 = 0 \\
	                           &                  &  &                                       &  & \lambda = -10
\end{align*}


\subparagraph{3)}

\begin{align*}
	 & Z = f(x, y) = \sqrt{1 - x^2 - y^2}                                                                                               \\
	 & L(x, y, \lambda) = \sqrt{1 - x^2 - y^2} + \lambda \left( {\left( x - \frac{1}{2} \right)}^2 + y^2 - \frac{1}{16} \right)         \\
	 & 1)\quad L_x = - \frac{\cancel{2} x}{\cancel{2} \sqrt{1 - x^2 - y^2}} + \lambda 2 \left( x - \frac{1}{2} \right) \overset{!}{=} 0 \\
	 & 2)\quad L_y = - \frac{\cancel{2} y}{\cancel{2} \sqrt{1 - x^2 - y^2}} + 2 y \lambda \overset{!}{=} 0                              \\
	 & \Rightarrow 2 y \lambda = \frac{y}{\sqrt{1 - x^2 - y^2}}                                                                         \\
	 & \text{(a)} \quad y = 0;\qquad \text{(b)} \quad \lambda = \frac{1}{\sqrt{1 - x^2 - y^2}}                                          \\
	 & 3)\quad R(x, y) = {\left( x - \frac{1}{2} \right)}^2 + y^2 - \frac{1}{16} = 0 \quad \text{\Lightning}
\end{align*}

\begin{align*}
	\underline{\text{(b) in } 1)}:                                                      \\
	- \frac{x}{\sqrt{1 - x^2 - y^2}} + \frac{x - \frac{1}{2}}{\sqrt{1 - x^2 - y^2}} = 0 \\ %BLITZ  \\
	\\
	\underline{\text{(a) in } 3)}:                                                      \\
	x^2 - x + \frac{3}{16} = 0                                                          \\
	x_1 = \frac{1}{4} \quad y_1 = 0                                                     \\
	x_2 = \frac{3}{4} \quad y_2 = 0
\end{align*}

\begin{align*}
	z   & = \sqrt{1 - x^2 - y^2}                            \\
	z^2 & = \sqrt{1 - x^2 - y^2}                            \\
	z^2 + x^2 + y^2 = 1 \Leftarrow \text{(Einheits-) Kugel} \\
\end{align*}
\[
	R(x,y) \Leftarrow \text{Kreis mit } r = \frac{1}{16} \text{ um } \frac{1}{2} \text{ in x-Richtung verschoben}
\]

% TODO: Add Skizze
