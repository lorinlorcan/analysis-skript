\subsection{Einleitung}

\begin{figure}[H]
    \centering
    \begin{tikzpicture}
        \begin{axis}[
                default,
                xmin=-0.5, xmax=4.5,
                ymin=-0.5, ymax=4.5,
                width=10cm,
                xticklabels={},
                yticklabels={}
            ]
            \addplot[orange, thick, smooth, samples=100, domain=-1:5] {-(0.5 * x - 2)^2 + 4};
            \addplot[mark=x] coordinates {(1, 1.75)};
            \addplot[mark=x] coordinates {(3, 3.75)};

            \draw[dotted, thick] (1, -0.2) -- (1, 1.75);
            \draw[dotted, thick] (3, -0.2) -- (3, 3.75);
            \draw[dotted, thick] (-0.2, 1.75) -- (1, 1.75);
            \draw[dotted, thick] (-0.2, 3.75) -- (3, 3.75);

            \node[below] at (axis cs: 1, -0.2) {\(x_0\)};
            \node[below] at (axis cs: 3, -0.2) {\(x_1\)};
            \node[left] at (axis cs: -0.2, 1.75) {\(y_0\)};
            \node[left] at (axis cs: -0.2, 3.75) {\(y_1\)};
            
            \draw[->] (1.5, 1.75) arc (0 : 58 : 0.4);
            \node[above right] at (axis cs: 1.15, 1.75) {\(\alpha\)};
            
            \draw (1, 1.75) -- (3, 1.75);
            \draw (3, 1.75) -- (3, 3.75);
            \draw[mkblue, thick] (0.5, 1.25) -- (3.5, 4.25);
            \node[mkblue, left] at (axis cs: 3, 4) {Sekante};
            \addplot[mkgreen, thick, smooth, samples=100, domain=0.5:2.2] {1.5*x+0.25};
            \node[mkgreen, left] at (axis cs: 2, 3.25) {Tangente in \(P_0\)};
            
            \node[below right] at (axis cs: 1, 1.75) {\(P_0\)};
            \node[below right] at (axis cs: 3, 3.75) {\(P_1\)};
            \node[below] at (axis cs: 2, 1.75) {\(\Delta x\)};
            \node[right] at (axis cs: 3, 2.75) {\(\Delta y\)};
        \end{axis}
    \end{tikzpicture}
    \caption{Graphische Herleitung des Differenzenquotient}
\end{figure}

\paragraph{Differenzenquotient}

\[
    \tan \alpha = \frac{\Delta y}{\Delta x} = \frac{y_1 - y_0}{x_1 - x_0}
\]

\paragraph{Differentialquotient}

\[
    \lim_{x_1 \rightarrow x_0} \frac{y_1 - y_0}{x_1 - x_0} = \tan \alpha_0
\]

\paragraph{1. Ableitung von \(f(x) \entspricht f'(x_0)\)}

\[
    \lim_{x_1 \rightarrow x_0} \frac{y_1 - y_0}{x_1 - x_0}= y'(x_0)
\]

\begin{align*}
    x_1 - x_0 &= \Delta x \\
    \Rightarrow x_1 &= \Delta x + x_0 \\
    \Rightarrow \lim_{x_1 \rightarrow x_0} \frac{y(x_1) + y(x_0)}{x_1 - x_0}
    &= \lim_{\Delta x \rightarrow 0} \frac{y(x_0 + \Delta x) - y(x_0)}{\Delta x}
\end{align*}

\paragraph{Ableitungsfolge}

\[
    \begin{array}{ccccccccccc}
        & \frac{\diff}{\diff x} &
        & \frac{\diff}{\diff x} &
        & \frac{\diff}{\diff x} &
        & \frac{\diff}{\diff x} & \\
        y(x)
        & \rightarrow & y'(x)
        & \rightarrow & y''(x)
        & \rightarrow & y'''(x)
        & \rightarrow & y^{(4)}(x)
        & \cdots \\
        & & \frac{\diff y}{\diff x}
        & & \frac{\diff^2 y}{\diff x^2}
        & & \frac{\diff^3 y}{\diff x^3}
        & & \frac{\diff^4 y}{\diff x^4}
    \end{array}
\]

\paragraph{Beispiele}

\subparagraph{Ableitung von Polynomfunktionen}

\begin{align*}
    y &= f(x) = x \quad y_1 = y(x_1) = x_1;\ y_2 = y(x_2) = x_2 \\
    y' &= \lim_{x_1 \rightarrow x_0} \frac{y_1 - y_0}{x_1 - x_0} \\
    &= \lim_{x_1 \rightarrow x_0} \frac{x_1 - x_0}{x_1 - x_0} = 1 \\
    \\
    y &= f(x) = x^2 \\
    y' &= \lim_{x_1 \rightarrow x_0} \frac{y_1 - y_0}{x_1 - x_0} \\
    &= \lim_{x_1 \rightarrow x_0} \frac{x_1^2 - x_0^2}{x_1 - x_0} \\
    &= \lim_{x_1 \rightarrow x_0} \frac{(x_1 - x_2)(x_1 + x_0)}{x_1 - x_0} \\
    &= \lim_{x_1 \rightarrow x_0} (x_1 + x_0) = 2x \\
    \\
    y &= f(x) = x^n \\
    y' &= \lim_{x_1 \rightarrow x_0} \frac{x_1^n - x_0^n}{x_1 - x_0}  &&\mid \text{Polynomdivision} \\
    &= \lim_{x_1 \rightarrow x_0} (x_0^0 \cdot x_1^{n-1} + x_0^1 \cdot x_1^{n-2} + x_0^2 \cdot x_2^{n-3} + \cdots + x_0^{n-1} \cdot x_1^0)\\
    &= n \cdot x^{n-1}
\end{align*}

\subparagraph{Ableitung von Logarithmusfunktionen}

\begin{align*}
    y &= f(x) = \log_a x \\
    y'(x_0) &= \lim_{\Delta x \rightarrow 0} \frac{\log_a(x_0 + \Delta x) - \log_a(x_0)}{\Delta x} \\
    &= \lim_{\Delta x \rightarrow 0} \frac{\log_a \left( \frac{x_0 + \Delta x}{x_0} \right)}{\Delta x} &&\mid \Delta x = \frac{x_0}{n} \\
    \Rightarrow &= \limtoinfty{n} \frac{\log_a \left( \frac{x_0 + \frac{x_0}{n}} {x_0} \right)}{\frac{x_0}{n}} \\
    &= \limtoinfty{n} n \cdot \frac{\log_a \left( 1 + \frac{1}{n} \right)}{x_0} \\
    &= \frac{1}{x_0} \limtoinfty{n} \log_a {\left( 1 + \frac{1}{n} \right)}^{n} \\
    &= \frac{1}{x_0} \log_a \left( \limtoinfty{n} {\left( 1 + \frac{1}{n} \right)}^n \right) &&\mid \text{Siehe Rechnung~\ref{eq:euler}} \\
    \Rightarrow y' &= \frac{1}{x} \cdot \log_a \e = \frac{1}{x \cdot \ln a} &&\mid \log_a \e = \frac{1}{\log_{\e} a} = \frac{1}{\ln a} \\
    \\
    y &= f(x) = \ln x = \log_{\e} x \\
    y' &= \frac{1}{x} \log_{\e} e = \frac{1}{x}
\end{align*}


\paragraph{Annäherung an die eulersche Zahl \( \e \)}
\begin{align}
    \begin{split}
        \label{eq:euler}
        \limtoinfty{n} a_n = \limtoinfty{n} {\left( 1 + \frac{1}{n} \right)}^n = e &\approx 2.718\dots \\
        a_n = {\left( 1 + \frac{1}{n} \right)}^n
        \Rightarrow a_1 &= 2 \\
        a_2 &= 2,25 \\
        a_3 &= 2,7 \\
        \vdots
    \end{split}
\end{align}
