\subsubsection{Kostenfunktion}

\begin{figure}[H]
	\centering
	\begin{tikzpicture}
		\begin{axis}[
				defaultpure,
				xmin=-0.5, xmax=3.5,
				ymin=-0.5, ymax=3.5,
				width=8cm,
				xlabel={x (Stück)}, ylabel={k},
				xticklabels={}, yticklabels={}
			]

			\addplot[orange, smooth, samples=100, domain=0:8] {0.5*(x-1)^3 + 1.5};
			\addplot[mkred, smooth, samples=100, domain=0:5] {(1.86)/(1.89) * x};


			\draw[dotted, thick] (1, 0) -- (1, 4);
			\draw[dotted, thick] (1.86, 0) -- (1.86, 6.51);

			\node[below] at (1, 0) {\( x_W \)};
			\node[below] at (1.86, 0) {\( x_D* \)};
		\end{axis}
	\end{tikzpicture}
	\caption{Skizze Kostenfunktion}
\end{figure}

\begin{tabular}{l l}
	\( \overline{k_D} = \frac{k}{x} \) & Durchschnittskosten/Stück \\
	\( k' = \frac{\diff k}{\diff x} \) & Grenzkosten               \\
	\( E(k,x) = \frac{k'}{\overline{k_D}} \)
\end{tabular}

\paragraph{Beispiel}

\begin{figure}[H]
	\centering
	\begin{tikzpicture}
		\begin{axis}[
				defaultpure,
				xmin=-100, xmax=2000,
				ymin=-100, ymax=100000,
				width=8cm,
				xlabel={p}, ylabel={x},
			]

			\addplot[orange, smooth, samples=100, domain=0:25000] {-60*x+90000};


			\draw[dotted, thick] (1, 0) -- (1, 4);
			\draw[dotted, thick] (1.86, 0) -- (1.86, 6.51);

			\node[below] at (1, 0) {\( x_W \)};
			\node[below] at (1.86, 0) {\( x_D* \)};
		\end{axis}
	\end{tikzpicture}
	\caption{Skizze Beispiel}
\end{figure}

\begin{alignat*}{2}
	x(p)                    & = 90\,000-60p\quad        &  & 1) p=500  \\
	\frac{\diff x}{\diff p} & = -60                     &  & 2) p=1000 \\
	x_1                     & = 90\,000 - 60 \cdot 500  &  & = 60\,000 \\
	x_2                     & = 90\,000 - 60 \cdot 1000 &  & = 30\,000
\end{alignat*}

\begin{align*}
	E(x, p)     & = \frac{\diff x}{\diff p} \cdot \frac{\diff p}{\diff x} \\
	E(x_1, p_1) & = -60 \cdot \frac{500}{60\,000} = -0,5                  \\
	E(x_2, p_2) & = -60 \cdot \frac{1000}{30\,000} = -2
\end{align*}

\paragraph{Polypol \& Monopol}

\begin{figure}[H]
	\centering
	\begin{tikzpicture}
		\begin{axis}[
				defaultpure,
				xmin=-0.5, xmax=3.5,
				ymin=-0.5, ymax=3.5,
				width=8cm,
				xlabel={x}, ylabel={G, E, K},
				xticklabels={},
				yticklabels={}
			]

			\addplot[name path=k, orange, smooth, samples=100, domain=0:4] {(x - 1.5)^3 + (x-2)^2 +1};
			\addplot[name path=e, mkred, smooth, samples=100, domain=0:4] {x};
			\addplot[mkblue, smooth, samples=100, domain=0:4] {1};

			\addplot fill between [of = k and e, split,
					every even segment/.style = {transparent},
					every odd segment/.style = {mkgreen!30}];

			\draw[dotted, thick] (1.38, 0) -- (1.38, 4);
			\draw[dotted, thick] (1.84, 0) -- (1.84, 4);
			\draw[dotted, thick] (2.58, 0) -- (2.58, 4);

			\node[below] at (1.38, 0) {\( G_S \)};
			\node[below] at (1.84, 0) {\( G_{Max} \)};
			\node[below] at (2.58, 0) {\( G_G \)};

			\node[above, mkblue] at (3, 1) {\( p(x) \)};
			\node[below, mkred] at (3, 3) {\( E \)};
			\node[above, orange] at (2.9, 3.2) {\( K \)};
		\end{axis}
	\end{tikzpicture}
	\quad
	\begin{tikzpicture}
		\begin{axis}[
				defaultpure,
				xmin=-0.5, xmax=3.5,
				ymin=-0.5, ymax=3.5,
				width=8cm,
				xlabel={x}, ylabel={G, E, K},
				xticklabels={},
				yticklabels={}
			]

			\addplot[name path=k, orange, smooth, samples=100, domain=0:4] {(x - 1.5)^3 + (x-2)^2 +1};
			\addplot[name path=e, mkred, smooth, samples=100, domain=0:4] {-(x - 1.4)^2 + 2};
			\addplot[mkblue, smooth, samples=100, domain=0:4] {-(2.2 / 2.81) * x + 2.2};

			\addplot fill between [of = k and e, split,
					every even segment/.style = {transparent},
					every odd segment/.style = {mkgreen!30}];

			\draw[dotted, thick] (1.02, 0) -- (1.02, 4);
			\draw[dotted, thick] (1.6, 0) -- (1.6, 4);
			\draw[dotted, thick] (2.19, 0) -- (2.19, 4);

			\node[below] at (1.02, 0) {\( G_S \)};
			\node[below] at (1.6, 0) {\( G_{Max} \)};
			\node[below] at (2.19, 0) {\( G_G \)};
			\node[below] at (2.8, 0) {\( \frac{P}{P_G} \)};

			\node[below, left] at (1.6, 0.9) {\( P_{opt} \)};
			\draw (1.5, 0.85) -- (1.7, 1.05);
			\draw (1.7, 0.85) -- (1.5, 1.05);

			\node[left, mkblue] at (0, 2.2) {\( P \)};
			\node[above, mkblue] at (3, 0) {\( p(x) \)};
			\node[below, mkred] at (3, 1) {\( E \)};
			\node[above, orange] at (2.9, 3.2) {\( K \)};
		\end{axis}
	\end{tikzpicture}
	\caption{Polypol (l.) vs Monopol (r.)}
\end{figure}

\subparagraph{Polypol}

Polypol \( \rightarrow \) Wettbewerb \( \rightarrow p(x) = \) konstant

\begin{align*}
	E   & = p(x) \cdot x                                 \\
	G   & = E - K                                        \\
	G'  & = E' - K' \overset{!}{=} 0 \Rightarrow E' = K' \\
	G_S & : \text{Gewinnspanne}                          \\
	G_G & : \text{Gewinngrenze}
\end{align*}

\subparagraph{Monopol}

Monopol \( \rightarrow \) Preisdiktatur \( \rightarrow p(x) = P - P_G(x) \)

\begin{align*}
	P       & : Prohibitivpreis                                               \\
	P_{opt} & : Gaurnotscher Punkt                                            \\
	E       & = p(x) \cdot x = Px - P_G \cdot x^2            & = x(p - P_G x) \\
	G       & = E - K                                                         \\
	G'      & = E' - K' \overset{!}{=} 0 \Rightarrow E' = K'
\end{align*}

\paragraph{Beispiel}

\begin{align*}
	K(x) & = {(x - 2)}^3 + 12 \\
	E(x) & = 12x
\end{align*}

a) Bei welchem \( x \) werden die Grenzkosten minimal? \\
b) Bei welchem \( x \) wird der Gewinn maximal?

\begin{align*}
	\text{a)}\quad K(x) & = {(x-2)}^3 + 12                           \\
	K'(x)               & = 3{(x-2)}^2 \leftarrow \text{Grenzkosten} \\
	K''(x)              & = 6(x-2)                                   \\
	\Rightarrow 0       & = 6(x-2)                                   \\
	6x - 12             & = 0                                        \\
	6x = 12                                                          \\
	x = 2                                                            \\
	\\
	K''(x)              & = 6 > 0 \Rightarrow \text{Minimum}
\end{align*}

\begin{align*}
	\text{b)}\quad G & = E - K                                  \\
	                 & = 12x - {(x-2)}^3 + 12                   \\
	\\
	G'               & = E'-K' \Rightarrow E' = K'              \\
	12 - 3{(x-2)}^2  & \overset{!}{=} 0                         \\
	12 - 3{(x-2)}^2  & = 0                                      \\
	12               & = 3{(x-2)}^2                             \\
	4                & = {(x-2)}^2                              \\
	2                & = x-2                                    \\
	4                & = x                                      \\
	\\
	G''(x)           & = -6(x-2)                                \\
	G''(x)           & = -6(x-2) < 0 \Rightarrow \text{Maximum}
\end{align*}
