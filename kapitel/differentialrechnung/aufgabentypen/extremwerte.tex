\subsubsection{Extremwerte}

\begin{figure}[H]
    \centering
    \begin{tikzpicture}
        \begin{axis}[
                default,
                xmin=-0.5, xmax=3.5,
                ymin=-0.5, ymax=3.5,
                width=8cm,
                xticklabels={},
                yticklabels={}
            ]
            \addplot[orange, smooth, samples=100, domain=-1:4] {(x-2)^3- 2 * x+6};
            
            \draw (0.7835034190722739, 3.0886621079036347) -- (1.583503419072274, 3.0886621079036347);
            \draw (2.4164965809277263, 0.9113378920963653) -- (3.2164965809277257, 0.9113378920963653);
            
            \draw[mkred, dotted, thick] (1.1835, -0.2) -- (1.1835, 4.2);
            \draw[mkred, dotted, thick] (2.8164, -0.2) -- (2.8164, 4.2);
        \end{axis}
    \end{tikzpicture}
    \qquad
    \begin{tikzpicture}
        \begin{axis}[
                default,
                xmin=-0.5, xmax=3.5,
                ymin=-0.5, ymax=3.5,
                width=8cm,
                xticklabels={},
                yticklabels={}
            ]
            \addplot[orange, smooth, samples=100, domain=-1:4] {(x-1.5)^2 + 1.5};
            
            \draw (1.1, 1.5) -- (1.9, 1.5);
            
            \draw[mkred, dotted, thick] (1.5, -0.2) -- (1.5, 4.2);
            
        \end{axis}
    \end{tikzpicture}
    \caption{Funktionen mit Extremwerten}
\end{figure}

\paragraph{Notwendige Bedingung}

\[
    y'(x) = 0
\]

\paragraph{Hinreichende Bedingung}

\[
    y^{(k)}(x_E)
    \begin{cases}
        < 0\ \text{Maximum} \\
        > 0\ \text{Minimum}    
    \end{cases}
    k \geq 2,\ \underbrace{\text{gerade, minimal}}_{\text{ungerade} \Rightarrow \text{Sattelpunkt}}
\]
