\subsubsection{Majoranten-/Miniorantenkriterium}

Gegeben sind die Reihen \( \sum\limits^{\infty}_{n=1} a_n \) und  \( \sum\limits^{\infty}_{n=1} b_n \), mit \( a_n \geq b_n \vee n > k (k \in \mathbb{N}) \), d.~h.

\begin{center}

	\( \sum\limits^{\infty}_{n=1} a_n \) ist eine \textit{Majorante} von \( \sum\limits^{\infty}_{n=1} b_n \)


	\medskip

	\( \sum\limits^{\infty}_{n=1} b_n \) ist eine \textit{Miniorante} von \( \sum\limits^{\infty}_{n=1} a_n \)

	\bigskip

	a) Es sei \( \sum\limits^{\infty}_{n=1} a_n \) konvergent \(\Rightarrow\) \( \sum\limits^{\infty}_{n=1} b_n \) ist konvergent

	\medskip

	b) Es sei \( \sum\limits^{\infty}_{n=1} b_n \) divergent \(\Rightarrow\) \( \sum\limits^{\infty}_{n=1} a_n \) ist divergent

\end{center}

\paragraph{Beispiel}

\begin{align*}
	1 +     & \frac{1}{2} + & \frac{1}{3} + & \frac{1}{4} + & \frac{1}{5} + & \frac{1}{6} + & \frac{1}{7} + & \frac{1}{8} + & \frac{1}{9} +  & \frac{1}{10} + & \frac{1}{11} + & \frac{1}{12} + & \frac{1}{13} + & \frac{1}{14} + & \frac{1}{15} + & \frac{1}{16} + & \cdots & = a_n \\
	\verteq & \verteq       & \vee          & \verteq       & \vee          & \vee          & \vee          & \verteq       & \vee           & \vee           & \vee           & \vee           & \vee           & \vee           & \vee           & \verteq        &        &       \\
	1 +     & \frac{1}{2} + & \frac{1}{4} + & \frac{1}{4} + & \frac{1}{8} + & \frac{1}{8} + & \frac{1}{8} + & \frac{1}{8} + & \frac{1}{16} + & \frac{1}{16} + & \frac{1}{16} + & \frac{1}{16} + & \frac{1}{16} + & \frac{1}{16} + & \frac{1}{16} + & \frac{1}{16} + & \cdots & = b_n \\
\end{align*}
\[
	\Rightarrow \text{divergent}
\]

