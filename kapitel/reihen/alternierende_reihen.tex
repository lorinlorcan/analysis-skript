\subsection{Alternierende Reihen}

\[
    a_0 - a_1 + a_2 - a_3 + a_4 - a_5 + a_6 - a_7 + a_8 - \ldots
\]

\subsubsection{Leibnizisches Konvergenzkriterium}

Wenn ihre Reihenglieder vom Betrag eine monoton abnehmende Nullfolge bildet, ist eine alternierende Reihe konvergent.

\[
    a_{n-2} > a_{n-1} > a_n > a_{n+1} > a_{n+2} \ldots  
\]

\paragraph{Beispiel}

\[
    \underbrace{\underbrace{\underbrace{1 - \frac{1}{2}}_{S_1} + \frac{1}{3}}_{S_2} - \frac{1}{4}}_{S_3} + \frac{1}{5} - \frac{1}{6} + \frac{1}{7} \ldots \quad \Rightarrow \text{monoton abnehmende Nullfolge} \Rightarrow \text{konvergent}
\]

\[
    \limtoinfty{n} S_n = ?
\]

\begin{align*}
    (&a_0 - a_1) + (a_2 - a_3) + (a_4 - a_5) + (a_6 - a_7) + \ldots \\
    &a_0 - (a_1 - a_2) - (a_3 - a_4) - (a_5 - a_6) - (a_7 - a_8) - \ldots
\end{align*}

\begin{align*}
    \left .
    \begin{array}{l}
        S_1 = (a_0 - a_1) \\
        S_3 = (a_0 - a_1) + (a_2 - a_3) \\
        S_5 = (a_0 - a_1) + (a_2 - a_3) + (a_4 - a_5) \\
        S_0 = a_0 \\
        S_2 = a_0 - (a_1 - a_2) \\
        S_4 = a_0 - (a_1 - a_2) - (a_3 - a_4)
    \end{array}
    \right \}
    \begin{array}{c}
        \text{nähern sich} \\
        \text{dem Summenwert}
    \end{array}
\end{align*}

\begin{figure}[H]
    \centering
    \begin{tikzpicture}
        \begin{axis}[
                default,
                xmin=-0.5, xmax=7.5,
                ymin=-0.5, ymax=4.5,
                width=10cm,
                xlabel={\( n \)},
                ylabel={\( S_n \)},
                xticklabels={},
                yticklabels={},
            ]
            \draw (0,2) -- (8,2);
            \node[left] at (0,2) {\( S \)};

            \coordinate[label=left:\( S_0 \)] (S0) at (0, 4);
            \coordinate[label=left:\( S_1 \)] (S1) at (1, 0.25);
            \coordinate[label=left:\( S_2 \)] (S2) at (2, 3.5);
            \coordinate[label=left:\( S_3 \)] (S3) at (3, 0.75);
            \coordinate[label=above left:\( S_4 \)] (S4) at (4, 3);
            \coordinate[label=left:\( S_5 \)] (S5) at (5, 1.25);
            \coordinate[label=left:\( S_6 \)] (S6) at (6, 2.5);
            \coordinate[label=left:\( S_7 \)] (S7) at (7, 1.75);

            \fill (S0) circle [radius=2pt];
            \fill (S1) circle [radius=2pt];
            \fill (S2) circle [radius=2pt];
            \fill (S3) circle [radius=2pt];
            \fill (S4) circle [radius=2pt];
            \fill (S5) circle [radius=2pt];
            \fill (S6) circle [radius=2pt];
            \fill (S7) circle [radius=2pt];
            
            \draw[decoration={brace, mirror, raise=5pt}, decorate]
            (4,3) -- node[left=6pt] {$F$} (4,2);
        \end{axis}
    \end{tikzpicture}
    \[
        F = | S - S_n | < a_n + 1  
    \]
    \caption{Alternierende Reihe -- die Werte nähern sich einem Summenwert \( S \)}
\end{figure}

\subsubsection{Alternierende harmonische Reihe}

\[
    \begin{array}{ccccc ccccc ccccc}
        S & = & 1 &- \frac{1}{2} &+ \frac{1}{3} &- \frac{1}{4} &+ \frac{1}{5} &- \frac{1}{6} &+ \frac{1}{7} &- \frac{1}{8} &+ \frac{1}{9} &- \frac{1}{10} &+ \frac{1}{11} &- \frac{1}{12} & \cdots \\
        \\
        \frac{1}{2}\ S & = & &+ \frac{1}{2} & &-\frac{1}{4} & &+ \frac{1}{6} & &- \frac{1}{8} & &+ \frac{1}{10} & &- \frac{1}{12} & \cdots \\
        \\
        \midrule
        \\
        \frac{3}{2}\ S & = & 1 & &+ \frac{1}{3} &- \frac{1}{2} &+ \frac{1}{5} & &+ \frac{1}{7} &- \frac{1}{4} &+ \frac{1}{9} & &+ \frac{1}{11} &- \frac{1}{6} & \cdots
    \end{array}
\]

\begin{tabular} {rl}
    alternierende harmonische Reihe: & (bedingt) konvergent \\
    harmonische Reihe: & divergent
\end{tabular}
 