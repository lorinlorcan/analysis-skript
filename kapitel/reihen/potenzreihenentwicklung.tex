\subsection{Potenzreihenentwicklung}

\begin{figure}[H]
    \centering
    \begin{tikzpicture}
        \begin{axis}[
                default,
                xmin=-0.5, xmax=4.5,
                ymin=-0.5, ymax=4.5,
                width=10cm,
                xticklabels={},
                yticklabels={}
            ]
            \addplot[mkblue, smooth, samples=100, domain=-1:4.5] {2};
            \addplot[mkblue, dashed, smooth, samples=100, domain=-1:4.5] { (x-2) + 2};
            \addplot[orange, smooth, samples=100, domain=-1:4.5] { (x-2)^2 + (x-2) + 2};
            \addplot[orange, smooth, samples=100, domain=-1:4.5] {(x-2)^3 + (x-2)^2 + (x-2) + 2};

            \node[left] at (-0.1, 2.2) {\( P_1 \)};
            \node[left] at (-0.1, 0.2) {\( P_2 \)};
            \node[left] at (-0.1, 3.8) {\( P_3 \)};
            \node at (0.8, -0.2) {\( P_4 \)};

            \draw[mkred, dotted, thick] (2, -1) -- (2, 5);
        \end{axis}
    \end{tikzpicture}
\end{figure}


\begin{align*}
    P(x) &= a_0 + a_1 {(x-x_0)}^1 + a_2 {(x-x_0)}^2 + a_3 {(x-x_0)}^3 + \cdots \\
    P_1 &= a_0 \\
    P_2 &= a_0 + a_1 (x-x_0) \\
    P_3 &= a_0 + a_1 (x-x_0) + a_2{(x-x_0)}^2 \\
    P_4 &= a_0 + a_1 (x-x_0) + a_2{(x-x_0)}^2 + a_3{(x-x_0)}^3 \\
\end{align*}

\[
\begin{array}{llccccc}
    P_n(x_0) &= f(x_0) = & a_0 \\
    P_n'(x_0) &= f'(x_0) = & a_1 + 2 a_2(x-x_0) &+ 3 a_3 {(x-x_0)}^2 &+ 4 a_4 {(x-x_0)}^3 &+ 5 a_5 {(x-x_0)}^4 &+ \cdots \\
    P_n''(x_0) &= f''(x_0) = & 2 a_2 & + 6 a_3 (x-x_0) &+ 4 \cdot 3 a_4 {(x-x_0)}^2 &+ 4 \cdot 5 a_5 {(x-x_0)}^3 &+ \cdots
\end{array}
\]

\begin{info}
\[
    P_n^n(x_0) = f^n (x_0) = 1 \cdot 2 \cdot 3 \cdot 4 \cdot 5 \cdot \cdots \cdot (n-1) \cdot n \cdot a_n
\]
\end{info}

\begin{align*}
    x &= x_0 \\
    P_n' (x_0) &= a_1 &\Rightarrow a_1 &= \frac{f'(x_0)}{1!} \\
    P_n'' (x_0) &= 2 a_2 &\Rightarrow a_2 &= \frac{f''(x_0)}{2!} \\ 
    P_n^n (x_0) & &\Rightarrow a_n &= \frac{f^n(x_0)}{n!} \\ 
\end{align*}

\[
    P(x) = \underbrace{\underbrace{f(x_0) + \frac{f'(x_0)}{1!} \cdot (x-x_0) + \frac{f''(x_0)}{2!}
    \cdot {(x-x_0)}^2 + \frac{f'''(x_0)}{3!} \cdot {(x-x_0)}^3 + 
    \cdots + \frac{f^n(x_0)}{n!} 
    \cdot {(x-x_0)}^n}_{\text{Taylor Polynom}} + \cdots}_{\text{Taylor Reihe}}
\]

\paragraph{Beispiel}

\begin{align*}
    y &= f(x) = e^x \\
    x_0 &= 0 \\
    \\
    f(x_0) = f(0) &= e^x &= 1 \\
    f'(0) &= e^0 &= 1 
\end{align*}

% TODO: Not happy about it 
\begin{align*}
    e^x &= P(x) = & 1 & &+ \frac{x}{1!} & &+ \frac{x^2}{2!} & &+ \frac{x^3}{3!} & &+ \frac{x^4}{4!} & &+ \cdots & &+ \frac{x^n}{n!} & &+ \cdots \\
    e^1 &= P(1) = & 1 & &+ \frac{1}{1!} & &+ \frac{1}{2!} & &+ \frac{1}{3!} & &+ \frac{1}{4!} & &+ \cdots & &+ \frac{1}{n!} & &+ \cdots \\
\end{align*}
