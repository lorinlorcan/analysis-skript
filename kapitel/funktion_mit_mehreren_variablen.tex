\section{Funktion mit mehreren Variablen}

\begin{align*}
	z & = f (x_1, x_2, x_3, \ldots)                                                                                                                                       \\
	z & = f (x, y)                                                                                                                                                        \\
	f & = \{ \lvert z (x, y) \rvert \quad (x, y) \in \underbrace{A \times B}_{\text{Definitionsbereich}},\quad z \in \underbrace{C}_{\text{Wertebereich}}, z = f(x, y) \}
\end{align*}

\paragraph{Beispiel}

\[
	z = x^2 + y^2 + 10
\]

\begin{figure}[H]
	\centering
	\input{kapitel/funktion_mit_mehreren_variablen/beispiel_3d_funktion.pgf}
	\caption{Die Funktion \(z = x^2 + y^2 + 10\)}
	Der gebildete Körper erscheint an den Seiten abgeschnitten um zu symbolisieren, dass es hier über die Achsen hinausgeht.
\end{figure}

\subsection{Potentielle Ableitungen}

% TODO: 3D-Graph

\begin{align*}
	y'                                & = \frac{\diff y}{\diff x}     \\
	y''                               & = \frac{\diff^2 y}{\diff x^2} \\
	\\
	\frac{\partial f}{\partial x}     & = f_x \qquad f_y              \\
	\frac{\partial^2 f}{\partial x^2} & =
	\begin{array}{ll}
		f_{xx} & f_{yx} \\
		f_{xy} & f_{yy}
	\end{array}                                         \\
	\\
	f_{yxy}                           & = f_{xyy} = f_{yyx}
\end{align*}

\begin{align*}
	z   & = f(x, y) = e^{x^2 + y^2}                                 \\
	f_x & = \left( x^2 + y^2 \right)' = e^{x^2 + y^2} \cdot 2 x
	    & f_{xy} = 2y \cdot e^{x^2 + y^2} \cdot 2x                  \\
	f_y & = \left( e^{x^2 + y^2} \right)' = e^{x^2 + y^2} \cdot 2 y
	    & f_{yx} = 2x \cdot e^{x^2 + y^2} \cdot 2y
\end{align*}

\paragraph{Übungen}

\subparagraph{1)}

\begin{align*}
	f(x, y) & = a^x \cdot y^a             \\
	f_x     & = a^x \cdot \ln a \cdot y^a \\
	f_y     & = a^x \cdot a \cdot y^{a-1} \\
	        & = a^{x+1} \cdot y^{a-1}
\end{align*}

\subparagraph{2)}

\begin{align*}
	f(x, y) & = e^{x y} \cdot \frac{x}{y}                                                               \\
	f_x     & = e^{x y} \cdot \frac{1}{y} + e^{x y} \cdot y \cdot \frac{x}{y}                           \\
	        & = e^{x y} \left( \frac{1}{y} + x \right)                                                  \\
	f_y     & = \frac{x}{y} \cdot e^{x y} \cdot x + e^{x y} \left(-x y^{-2} \right)                     \\
	        & = e^{x y} \left( \frac{x^2}{y} - \frac{x}{y^2} \right)                                    \\
	f_{xy}  & = e^{x y} \cdot x \left( \frac{1}{y} + x \right) + e^{x y} \left( - \frac{1}{y^2} \right) \\
	        & = e^{x y} \left( \frac{x}{y} + x^2 - \frac{1}{y^2} \right)
\end{align*}

\subsection{Extremwerte mit k ohne Nebenbedingung}

\begin{align*}
    \frac{\partial f}{\partial x} &= 0 \\
    \frac{\partial f}{\partial y} &= 0 \\
\end{align*}

\paragraph{Hinreichende Bedingung}

% TODO: Korrigieren!

\begin{align*}
    f_{xx} f_{yy} - f_{(xy)^2} &> 0 \Rightarrow \text{Extremwert} \\
    &< 0 \Rightarrow \text{Sattelpunkt}
\end{align*}

\begin{align*}
    \begin{rcases}
        f_{xx} (P) &< 0 \\
        \text{oder}\quad f_{yy} (P) &< 0
    \end{rcases}
    &\text{Maximum} \\
    \begin{rcases}
        f_{xx} (P) &> 0 \\
        \text{oder}\quad f_{yy} (P) &> 0
    \end{rcases}
    &\text{Minimum}
\end{align*}

\paragraph{Beispiel}

\subparagraph{1)}

\begin{align*}
    z &= f(x, y) = 2x + y - \frac{x^2 + y^2}{4} \\
    f_x &= 2 - \frac{x}{2} \overset{!}{=} 0 \Rightarrow x_0 = 4 \\
    f_z &= 1 - \frac{y}{2} \overset{!}{=} 0 \Rightarrow y_0 = 2 \\
    \\
    R &= \{ (x, y) x^2 + y^2 - 9 = 0 \} \\
    & 0 \leq x, y \leq 3
\end{align*}

\begin{align*}
    L(x, y, \lambda) &= f(x, y) + \lambda \cdot R(x, y) \\
    &= 2x + y - \frac{x^2 + y^2}{4} + \lambda(x^2 + y^2 - 9) \\
    \begin{rcases}
        \frac{\partial L}{\partial x} &\overset{!}{=} 0 \\
        \frac{\partial L}{\partial y} &\overset{!}{=} 0 \\
        \frac{\partial L}{\partial \lambda} &\overset{!}{=} 0
    \end{rcases}
    &x, y, \lambda \\
    x &= \frac{2}{\frac{1}{x} - 2 \lambda} \\
    y &= \frac{2}{\frac{1}{2} + 2 \lambda}
\end{align*}

\subparagraph{2)}

\begin{align*}
    U(x, y) = 2 \cdot x \cdot y \\
    p_x = 3, p_y = 2 \qquad &C = 60 = p_x x + p_y y = 3x + 2y \\
    &\Rightarrow R = 3x + 2y - 60
\end{align*}
\begin{align*}
    L(x, y, \lambda) = 2xy + \lambda (3x + 2y - 60) \\
    L_x = 2 y + 3 \lambda &\overset{!}{=} 0 &&\Rightarrow y = - \frac{3}{2} \lambda &&\Rightarrow y = 15 \\
    L_y = 2 x + 3 \lambda &\overset{!}{=} 0 &&\Rightarrow x = - \lambda  &&\Rightarrow x = 10 \\
    L_\lambda = 3 x + 2 y - 60 &\overset{!}{=} 0 && &&\Rightarrow - 3 \lambda - 3 \lambda - 60 = 0 \\
    & && &&\lambda = -10
\end{align*}


\subparagraph{3)}

\begin{align*}
    & Z = f(x, y) = \sqrt{1 - x^2 - y^2} \\
    & L(x, y, \lambda) = \sqrt{1 - x^2 - y^2} + \lambda \left( \left( x - \frac{1}{2} \right)^2 + y^2 - \frac{1}{16} \right) \\
    &1)\quad L_x = - \frac{\cancel{2} x}{\cancel{2} \sqrt{1 - x^2 - y^2}} + \lambda 2 \left( x - \frac{1}{2} \right) \overset{!}{=} 0 \\
    &2)\quad L_y = - \frac{\cancel{2} y}{\cancel{2} \sqrt{1 - x^2 - y^2}} + 2 y \lambda \overset{!}{=} 0 \\
    & \Rightarrow 2 y \lambda = \frac{y}{\sqrt{1 - x^2 - y^2}} \\
    & \text{(a)} \quad y = 0;\qquad \text{(b)} \quad \lambda = \frac{1}{\sqrt{1 - x^2 - y^2}} \\
    &3)\quad R(x, y) = \left( x - \frac{1}{2} \right)^2 + y^2 - \frac{1}{16} = 0 \quad \text{\Lightning}
\end{align*}

\begin{align*}
    \underline{\text{(b) in } 1)}: \\
    - \frac{x}{\sqrt{1 - x^2 - y^2}} + \frac{x - \frac{1}{2}}{\sqrt{1 - x^2 - y^2}} = 0 \\ %BLITZ  \\
    \\
    \underline{\text{(a) in } 3)}: \\
    x^2 - x + \frac{3}{16} = 0 \\
    x_1 = \frac{1}{4} \quad y_1 = 0 \\
    x_2 = \frac{3}{4} \quad y_2 = 0
\end{align*}

\begin{align*}
    z &= \sqrt{1 - x^2 - y^2} \\
    z^2 &= \sqrt{1 - x^2 - y^2} \\
    z^2 + x^2 + y^2 = 1 \Leftarrow \text{(Einheits-) Kugel} \\
    \\
    R(x,y) &\Leftarrow \text{Kreis mit } r = \frac{1}{16} \text{ um } \frac{1}{2} \text{ in x-Richtung verschoben} 
\end{align*}

% TODO: Add Skizze

