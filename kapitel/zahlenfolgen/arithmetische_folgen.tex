\subsection{Arithmetische Folgen}

\subsubsection{Bildungsgesetz}

\begin{align*}
	a_n     & = \frac{a_{n+1} + a_{n-1}}{2} \\
	2 a_n   & = a_{n+1} + a_{n-1}           \\
	a_{n+1} & = 2 a_n - a_{n-1}             
\end{align*}

\begin{gesetz}
	\paragraph{Rekursives Bildungsgesetz}
	
	\[
		a_{n+1} = 2 a_n - a_{n-1}
	\]
\end{gesetz}

\paragraph{Beispiel}

\subparagraph{Anfangsbedingung:}

\begin{align*}
	a_1 & = a     \\
	a_2 & = a + d 
\end{align*}

\[
	\begin{alignedat}{4}
		a_3 & : a_{2+1} & = 2a_2 - a_1 & \Rightarrow a_3 = 2(a+d) - a       & = a + 2d \\
		a_4 & : a_{3+1} & = 2a_3 - a_2 & \Rightarrow a_4 = 2(a+2d) - (a+d)  & = a + 3d \\
		a_5 & : a_{4+1} & = 2a_4 - a_3 & \Rightarrow a_4 = 2(a+3d) - (a+2d) & = a + 4d 
	\end{alignedat}
\]


\begin{gesetz}
	\paragraph{Explizites Bildungsgesetz}
	
	\[
		a_n = a + (n - 1)d
	\]
\end{gesetz}

\subsubsection{Beispiel für eine Folge 1. Ordnung}

\[
	\langle a_n \rangle = 6 \underbrace{,}_{4} 10 \underbrace{,}_{4} 14 \underbrace{,}_{4} 18,\ \ldots \\
\]

\paragraph{Allgemeine Lösung}

\[
	a_n = c_1 \cdot n + c_2 \\
\]

\paragraph{Lösungsweg}

\begin{align*}
	\left.
		\begin{array}{l}
			6 = c_1 \cdot 1 + c_2 \\
			10 = c_1 \cdot 2 + c_2
		\end{array}
	\right \}
	\left.
		\begin{array}{l}
			c_1 = 4 \\
			c_2 = 2
		\end{array}
	\right.
\end{align*}

\paragraph{Lösung}

\[
	*a_n = 6 + (n - 1) \cdot 4 = 2 + 4 n
\]

\subsubsection{Beispiel für eine Folge 2. Ordnung}

\[
	\langle a_n \rangle = 1,\ 7,\ 17,\ 31,\ 49,\ \ldots 
\]

\[
	\begin{tabular}{ccccccccccccc}
		1&,&7&,&17&,&31&,&49\\[-3pt]
		& \ubf&&\ubf&&\ubf&&\ubf \\
		& 6&&10&&14&&18\\[-3pt]
		&  &\ubf&&\ubf&&\ubf \\
		&  &4&&4&&4
  	\end{tabular}
\]

\paragraph{Allgemeine Lösung}

\[
	a_n = c_1\ n^2 + c_2\ n + c_3
\]

\paragraph{Lösungsweg}

\[
	\left.
		\begin{array}{lcl}
			1 & = c_1 \cdot 1^2 + c_2 \cdot 1 + c_3 \\
			7 & = c_1 \cdot 2^2 + c_2 \cdot 2 + c_3 \\
			17 & = c_1 \cdot 3^2 + c_2 \cdot 3 + c_3
		\end{array}
	\right \}
	a_n = 2n^2 {-} 1
\]
