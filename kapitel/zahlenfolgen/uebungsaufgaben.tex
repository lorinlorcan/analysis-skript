\subsection{Übungsaufgaben}

\begin{uebung}[1]
    \begin{aufgabe}
        \[
            \langle a_n \rangle = 1,\ 6,\ 11,\ 16,\ 21,\ \ldots
        \]
    \end{aufgabe}

    \begin{loesung}
        \[
            1\underbrace{\quad}_{5}6\underbrace{\quad}_{5}11\underbrace{\quad}_{5}16\underbrace{\quad}_{5}21 
        \]
        \[
            a_n = 1+(5n - 1) = 5n - 4
        \]
    \end{loesung}    
\end{uebung}

\begin{uebung}[2]
    \begin{aufgabe}
        \[
            \langle a_n \rangle = 5,\ 11,\ 21,\ 35,\ 53,\ \ldots
        \]
    \end{aufgabe}

    \begin{loesung}
        \[
            \begin{tabular}{ccccccccccccc}
                5&&11&&21&&35&&53\\[-3pt]
                & \ubf&&\ubf&&\ubf&&\ubf \\
                & 6&&10&&14&&18\\[-3pt]
                &  &\ubf&&\ubf&&\ubf \\
                &  &4&&4&&4
            \end{tabular}
        \]
        \[
            a_n = c_1 \cdot n^2 + c_2 \cdot n + c_3 \\
        \]
        \begin{align*}
            \left.
            \begin{array}{ccccc}
                5 &= 1 c_1  &+ 1 c_2 &+ c_3 \\
                11 &= 4 c_1 &+ 2 c_2 &+ c_3 \\
                21 &= 9 c_1 &+ 3 c_2 &+ c_3
            \end{array}
            \right \}
            \left.
            \begin{array}{cc}
                c_1 &= 2 \\
                c_2 &= 0 \\
                c_3 &= 3
            \end{array}
            \right \}
            a_n = 2n^2 + 3
        \end{align*}
    \end{loesung}    
\end{uebung}


\begin{uebung}[3]
    \begin{aufgabe}
        \[
            \langle a_n \rangle = \frac{2}{1!},\ \frac{5}{2!},\ \frac{8}{3!},\ \frac{11}{4!},\ \frac{14}{5!},\ \ldots
        \]
    \end{aufgabe}

    \begin{loesung}
        \subparagraph{Nenner}
        \[
            a_n = \frac{x}{n!}  
        \]
        \subparagraph{Zähler}
        \[
            \begin{tabular}{ccccccccccccc}
                & 3 && 3 && 3 && 3 \\[-3pt]
                & \obf && \obf && \obf && \obf \\
                2 && 5 && 8 && 11 && 14 \\
                \cline{1-1} \cline{3-3} \cline{5-5} \cline{7-7} \cline{9-9}\\[-10pt]
                1! && 2! && 3! && 4! && 5! \\
            \end{tabular}
        \]
        \[
            x = 2 + (n-1) 3 = 3n - 1
        \]
        \[
            a_n = \frac{3n - 1}{n!}
        \]
    \end{loesung}    
\end{uebung}


