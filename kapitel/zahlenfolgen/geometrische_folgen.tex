\subsection{Geometrische Folgen}

\subsubsection{Bildungsgesetz}

\begin{align*}
    a_n &= \sqrt{a_{n+1} \cdot a_{n-1}} \\
    a_n^2 &= a_{n+1} \cdot a_{n-1}
\end{align*}

\begin{gesetz}
    \paragraph{Rekursives Bildungsgesetz}
    \[
        a_{n+1} = \frac{a_n^2}{a_{n-1}}  
    \]
\end{gesetz}

\begin{align*}
    a_1 &= a \\
    a_2 &= a \cdot q \\
    a_3 &= \frac{{(a \cdot q)}^2}{a} = a \cdot q^2 \\
    a_4 &= \frac{{(a \cdot q^2)}^2}{a \cdot q} = a \cdot q^3
\end{align*}

\begin{gesetz}
    \paragraph{Explizites Bildungsgesetz}
    \[
        a_n = a \cdot q^{n-1}  
    \]
\end{gesetz}

\[
    q = \frac{a_{n+1}}{a_n}
\]

\subsubsection{Summe einer geometrischen Folge}

\[
    \begin{array}{ccccccccc}
        q \cdot S_n &=& & a \cdot q^1 &+ a \cdot q^2 &+ a \cdot q^3 &+ \cdots &+ a \cdot q^{n-1} &+ a \cdot q^n \\
        S_n &=& a &+ a \cdot q^1 &+ a \cdot q^2 &+ a \cdot q^3 &+ \cdots &+ a \cdot q^{n-1} & \\
        \cline{1-9}
        q \cdot S_n - S_n &=& -a &+ a \cdot a^n &&&&& \\
        S_n(q-1) &=& a(q^n -1) &&&&&& \\
        S_n &=& a \cdot \frac{q^n - 1}{q - 1} &&&&&& \\
        &=& a \cdot \frac{(q^n - 1)(-1)}{(q - 1)(-1)} &&&&&& \\
        &=& a \cdot \frac{1- q^n}{1 - q} &&&&&& \\
    \end{array}  
\]

\begin{gesetz}
    \paragraph{Summe einer geometrischen Folge bis \(n\)}
    \[
        S_n = a \cdot \frac{1 - q^n}{1 - q}
    \]
\end{gesetz}

\paragraph{1. Fall: \(q > 1, a > 0\)}

\[
    S_n \rightarrow \infty    
\]

\paragraph{2. Fall: \(0 < q < 1, a > 0\)}

\[
    S_n = a \cdot \frac{1}{1-q} 
\]
