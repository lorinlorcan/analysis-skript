\subsection{Rekursionsschnecke}

\begin{figure}[H]
	\centering
	\begin{tikzpicture}[scale=0.6]
		% Achsen
		\draw[dashed] (-12, 0) -- (6, 0);
		\node[right] at(6, 0) {\( G_1 \)};
		
		\draw[dashed] (-6, 6) -- (6, -6);
		\node[below right] at (6, -6) {\( G_2 \)};
		
		\draw[dashed] (0, 6) -- (0, -6);
		\node[below] at (0, -6) {\( G_3 \)};
		
		\draw[dashed] (-6, -6) -- (6, 6);
		\node[below left] at (-6, -6) {\( G_4 \)};
		
		\draw[color1, thick] (-10, 0) -- (0, 0);
		\node[color1, below] at (-5, 0) {\( D = 10 \)};
		\node[right] at (0.25, 0.25) {\( M \)};
		\node[below] at (-10, 0) {\( A_1 \)};
		\node[left] at (-7.5, 2.5) {\( a_1 \)};
		
		\draw (-10, 0) -- (-5, 5);
		\node[above] at (-5, 5) {\( A_2 \)};
		\node[above] at (-2.5, 5) {\( a_2 \)};
		
		\draw (-5, 5) -- (0, 5);
		\node[right] at (0, 5) {\( A_3 \)};
		\node[right] at (1.25, 3. 75) {\( a_3 \)};
		
		\draw (0, 5) -- (2.5, 2.5);
		\node[right] at (2.5, 2.5) {\( A_4 \)};
		\node[right] at (2.5, 1.25) {\( a_4 \)};
		
		% a_5 bis ...
		\draw (2.5, 2.5) -- (2.5, 0);
		\draw (2.5, 0) -- (1.25, -1.25);
		\draw (1.25, -1.25) -- (0, -1.25);
		\draw (0, -1.25) -- (-0.625, -0.625);
		\draw (-0.625, -0.625) -- (-0.625, 0);
		\draw (-0.625, 0) -- (-0.3125, 0.3125);
		\draw (-0.3125, 0.3125) -- (0, 0.3125);
		\draw (0, 0.3125) -- (0.15625, 0.15625);
		\draw (0.15625, 0.15625) -- (0.15625, 0);
		\draw (0.15625, 0) -- (0.078125, -0.078125);
		\draw (0.078125, -0.078125) -- (0, -0.078125);
		
		% color2
		\draw[color2, thick] (-5, 5) -- (0, 0);
		\node[color2, right] at (-2.5, 2.5) {\( b_1 \)};
		
		% Winkel
		\draw (-9, 0) arc (0:45:1);
		\node at (-9.4, 0.25) {\( \alpha \)};
		
		\draw (-4.55, 4.55) arc (-30:-150:0.5);
		\node at (-5, 4.6) [circle, fill, inner sep=0.5pt] {};
		
		\draw (0, 4.5) arc (-90:-180:0.5);
		\node at (-0.2, 4.8) [circle, fill, inner sep=0.5pt] {};
		
		\draw (2.15, 2.15) arc (-130:-220:0.5);
		\node at (2.2, 2.5) [circle, fill, inner sep=0.5pt] {};
		
	\end{tikzpicture}
	\caption{Die \enquote{Rekursionsschnecke}}\label{fig:rekusionsschnecke}
\end{figure}

\paragraph{Aufgabe 1}

\[
	a_n =\ ?
\]

\paragraph{Lösungsweg}

\subparagraph{Satz des Pythagoras}

\begin{align*}
	a_1^2 + a_1^2 & = D^2       \\
	2 a_1^2       & = 10^2      \\
	a_1^2         & = 50        \\
	a_1           & = \sqrt{50}
\end{align*}

\begin{align*}
	a_2^2 + a_2^2 & = a_1^2                                                    \\
	2 a_2^2       & = {\sqrt{50}}^2                                            \\
	a_2^2         & = \frac{50}{2}                                             \\
	a_2           & = \sqrt{\frac{50}{2}} = \sqrt{50} \cdot \frac{1}{\sqrt{2}}
\end{align*}

\begin{align*}
	a_3^2 + a_3^2 & = a_2^2                              \\
	2 a_3^2       & = \frac{50}{2}                       \\
	a_3^2         & = \frac{50}{4}                       \\
	a_3           & = \sqrt{50} \cdot \frac{1}{\sqrt{4}}
\end{align*}

\[
	\langle a_n \rangle = \sqrt{50},\ \sqrt{50} \cdot \frac{1}{\sqrt{2}},\ \sqrt{50} \cdot \frac{1}{\sqrt{4}},\ \ldots \
\]

\subparagraph{geometrische Folge}

\begin{align*}
	a_n           & = a \cdot q^{n-1}                                         \\
	a_n           & = \sqrt{50} \cdot \sqrt{\frac{1}{2^{n-1}}}                \\
	              & = \sqrt{50} \cdot {\left(\frac{1}{\sqrt{2}}\right)}^{n-1} \\
	\Rightarrow q & = \frac{1}{\sqrt{2}}
\end{align*}

\paragraph{Aufgabe 2}

\[
	S =\ ?
\]

\paragraph{Lösungsweg}

\[
	S = a \cdot \frac{1 - q^n}{1 - q} = \sqrt{50} \cdot \frac{1}{1- \frac{1}{\sqrt{2}}} \approx 24,14
\]
