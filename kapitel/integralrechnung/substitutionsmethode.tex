\subsection{Substitutionsmethode}

\[
    \int \frac{1}{x+2} \;\mathrm{d}x = \int \frac{1}{z} \;\mathrm{d}z = \ln z + C = \ln (x+z) + C
\]

\begin{align*}
    z = x + 2 &\Rightarrow x = z - 2 \\
    z' = \frac{\mathrm{d}z}{\mathrm{d}x} &\Rightarrow \mathrm{d}z = \mathrm{d}x
\end{align*}

\begin{align*}
    \int_2^3 \frac{1}{x+2} \;\mathrm{d}x &= \int_4^5 \frac{1}{z} \;\mathrm{d}z = {[\ln(z)]}_4^5 \\
    &\text{oder} \\
    &= \int_{z_n}^{z_0} \frac{1}{2} \;\mathrm{d}z = {[\ln(x+2)]}_2^3
\end{align*}

\paragraph{Beispiel}

% TODO: Hier und andere Stellen davor: Heißt es z_0 oder z_o ?

\begin{align*}
    &\int_0^1 \frac{1-x}{\sqrt{2x-x^2}} \;\mathrm{d}x
    \qquad z = 2x - x^2 \\
    \\
    &\qquad \frac{\mathrm{d}z}{\mathrm{d}x} = 2 - 2x = (1-x)2
    \qquad \frac{\mathrm{d}z}{2(1-x)} = \mathrm{d}x \\
    \\
    &\Rightarrow - \int_{z_n}^{z_0} \frac{1-x}{\sqrt{z}} - \frac{\mathrm{d}z}{2(1-x)}
    = \frac{1}{2} \int_{z_n}^{z_0} \frac{1}{\sqrt{2}} \mathrm{d}z \\
    &= \frac{1}{2} \int_{z_n}^{z_0} z^{-\frac{1}{2}} \mathrm{d}z 
    = \frac{1}{2} {\left[ \frac{z^{\frac{1}{2}}}{\frac{1}{2}} \right]}_{z_n}^{z_0}
    = {\left[ {\left( 2x-x^2 \right)}^\frac{1}{2} \right]}_0^1 \\
    &\text{oder} \\
    &\Rightarrow = {\left[ \sqrt{z} \right]}_0^1 = 1
\end{align*}
