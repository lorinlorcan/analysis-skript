\subsection{Komplexe Zahlen}

Die Gleichung \( x^2 = -1 \) hat keine Lösung in \( \mathbb{R} \), da kein Quadrat einer reellen Zahlen negativ sein kann.

Einführung der \enquote{imaginären Zahl} \( i = \sqrt{-1} \)

\[
	\mathbb{C} = \{ z=a+ib \mid a,b \in \mathbb{R} \}
\]

Imaginäre Zahlen setzen sich aus einem Imaginärteil und einem Realteil zusammen und werden in der \textit{Gaußschen Zahlenebene} (siehe Abb.~\ref{fig:gausche_zahlenebene}) dargestellt.

\begin{figure}
	\centering
	\begin{tikzpicture}
		\begin{axis}[
				defaultnonumbers,
				xmin=-1.2, xmax=5.2,
				ymin=-3.2, ymax=3.2,
				width=8cm,
				xlabel={\( \mathrm{Re}(z) \)},
				ylabel={\( \mathrm{Im}(z) \)}
			]
			\addplot[mkblue, thick, mark=x] coordinates {(2, 2)};
			\addplot[mkblue, thick, mark=x] coordinates {(2, -2)};

			\draw (2, -0.1) -- (2, 0.1);
			\draw (-0.1, 2) -- (0.1, 2);
			\draw (-0.1, -2) -- (0.1, -2);

			\node[left] at (axis cs: 0, 2) {\( b \)};
			\node[left] at (axis cs: 0, -2) {\( -b \)};
			\node[below] at (axis cs: 2, 0) {\( a \)};

			\draw[mkblue, dashed, thick] (0, 0) -- (2, 2);
			\draw[mkblue, dashed, thick] (0, 0) -- (2, -2);

			\node[above right] at (axis cs: 2, 2) {\( z=a+ib \)};
			\node[below right] at (axis cs: 2, -2) {\( z=a-ib \)};
		\end{axis}
	\end{tikzpicture}
	\caption{Die Gausche Zahlenebene}\label{fig:gausche_zahlenebene}
\end{figure}

\subsubsection{Algebraische Darstellungsform}


\begin{align*}
	z_1 & = a_1 + i b_1 \\
	z_2 & = a_2 + i b_2
\end{align*}


\subsubsection{Addition}


\begin{align*}
	z_3 & = z_1 + z_2                                       \\
	    & = (a_1 + i b_1) + (a_2 + i b_2)                   \\
	    & = a_1 + i b_1 + a_2 + i b_2                       \\
	    & = \underbrace{(a_1 + a_2)}_{a_3 \in \mathbb{R}} +
	\underbrace{i(b_1 + b_2)}_{b_3 \in \mathbb{C}}
\end{align*}

\subsubsection{Multiplikation}

\begin{align*}
	z_3 & = z_1 \cdot z_2                                         \\
	    & = (a_1 + i b_1) \cdot (a_2 + i b_2)                     \\
	    & = a_1 a_2 + a_1 ib_2 + ib_1 a_2 + ib_1 ib_2             \\
	    & = \underbrace{a_1 a_2 - b_1 b_2}_{a_3 \in \mathbb{R}} +
	\underbrace{i(a_1 b_2 + a_2 b_1)}_{b_3 \in \mathbb{C}}
\end{align*}


\subsubsection{Trigonometrische Darstellungsform}

\begin{figure}
	\centering
	\begin{tikzpicture}
		\begin{axis}[
				defaultnonumbers,
				xmin=-1.2, xmax=5.2,
				ymin=-3.2, ymax=3.2,
				width=8cm,
				xlabel={\( \mathrm{Re}(z) \)},
				ylabel={\( \mathrm{Im}(z) \)}
			]
			\addplot[mkblue, thick, mark=x] coordinates {(2, 2)};
			\addplot[mkblue, thick, mark=x] coordinates {(2, -2)};

			\draw (2, -0.1) -- (2, 0.1);
			\draw (-0.1, 2) -- (0.1, 2);
			\draw (-0.1, -2) -- (0.1, -2);

			\node[left] at (axis cs: 0, 2) {\( b \)};
			\node[left] at (axis cs: 0, -2) {\( -b \)};
			\node[below] at (axis cs: 2, 0) {\( a \)};

			\draw[mkblue, dashed, thick] (0, 0) -- (2, 2);

			\draw[mkred, dashed, thick] (0, 2) -- (2, 2);
			\draw[mkgreen, dashed, thick] (2, 0) -- (2, 2);

			\node[above right] at (axis cs: 2, 2) {\( z=a+ib \)};
			\node[mkgreen, right] at (axis cs: 2, 1) {\( r \cdot \sin \varphi \)};
			\node[mkred, above] at (axis cs: 1, 2) {\( r \cdot \cos \varphi \)};
			\node[below right] at (axis cs: 2, -2) {\( z=a-ib \)};
			\node[below] at (axis cs: 0.6, 0.6) {\( \varphi \)};
			\node[mkblue, above, rotate=45] at (axis cs: 1, 1) {\( |z|=r \)};

			\draw[->] (1, 0) arc (0 : 45 : 1);
		\end{axis}
	\end{tikzpicture}
	\caption{Die trigonometrische Darstellungsform von \( \mathbb{C} \)}\label{fig:trigonometrische_form}
\end{figure}

\[
	\begin{alignedat}{3}
		\vert z \vert &=\sqrt{a^2+b^2} \\
		\cos(\phi) &= \frac{a}{r} && \sin(\phi) &= \frac{b}{r} \\
		a &= r \cdot \cos(\phi) && r \cdot \sin(\phi) &= b \\
		z &= r \cos(\phi) + i r \sin(\phi) \\
		&= r (\cos(\phi) + i \sin(\phi))
	\end{alignedat}
\]

Vergleichend dazu siehe Abb.~\ref{fig:trigonometrische_form}.

\subsubsection{Eulersche Schreibweise}


\begin{align*}
	z & = r (\cos(\phi)) + i \cdot \sin(\phi) \\
	  & = r \cdot e^{i\ \phi}
\end{align*}

\subsubsection{Eulersche Gleichung}

\[
	\cos(\phi) + i \sin(\phi) = e^{i\ \phi}
\]

\paragraph{Multiplikation}

\begin{align*}
	z_1 & = r_1 \cdot e^{i\ \phi_1}                                                              \\
	z_2 & = r_2 \cdot e^{i\ \phi_2}                                                              \\
	\\
	z_3 & = z_1 \cdot z_2                                                                        \\
	    & = r_1 \cdot e^{i\ \phi_1} \cdot r_2 \cdot e^{i\ \phi_2}                                \\
	    & = \underbrace{r_1 r_2}_{r_3} \cdot \underbrace{e^{i(\phi_1 + \phi_2)}}_{e^{i\ \phi_3}}
\end{align*}

\begin{uebung}
	\begin{aufgabe}
		\[
			z^3 = 1
		\]
	\end{aufgabe}
	\begin{loesungsweg}
		\[
			\text{Es gilt:}
			r \cdot e^{i\ \phi} = z
		\]

		\subparagraph{Berechnung von \( z_1 \):}


		\begin{align*}
			z^3 & = {\left(r \cdot e^{i\ \phi}\right)}^3                                           \\
			z^3 & = r^3 \cdot e^{3 i\ \phi}                                                        \\
			1   & = r^3 \cdot e^{3 i\ \phi}                                                        \\
			1   & = r(\underbrace{\cos{(3 \phi)}}_{=1} + \underbrace{i \cdot \sin{(3 \phi)}}_{=0})
		\end{align*}

		\subparagraph{Berechnung von \( z_2 \) und \( z_3 \):}

		\[
			\begin{alignedat}{1}
				\cos{(3 \phi)} &= 1 \\
				3 \phi &= 2 \pi \\
				\phi &= \frac{2}{3} \pi \\
				\text{Es gilt aber auch:} \\
				3 \phi &= 4 \phi \\
				\phi &= \frac{4}{3} \pi
			\end{alignedat}
		\]
	\end{loesungsweg}

	\begin{loesung}
		\[
			\begin{alignedat}{3}
				z_1    &     & z_2       &                   & z_3       &                   \\
				r_1    & = 1 & \quad r_2 & = 1               & \quad r_3 & = 1               \\
				\phi_1 & = 0 & \phi_2    & = \frac{2}{3} \pi & \phi_3    & = \frac{4}{3} \pi
			\end{alignedat}
		\]
	\end{loesung}
\end{uebung}
