% Box für Übungen
\newenvironment{uebung}{
	\begin{tcolorbox}[sharp corners, colback=white, colframe=mkblue, title=\textbf{Übung}]
		}
		{
	\end{tcolorbox}
}

% Box für Gesetze, die hervorgehoben werden sollen
\newenvironment{gesetz}{
	\begin{tcolorbox}[sharp corners, colback=white, colframe=mkred]
		}
		{
	\end{tcolorbox}
}

% Box für ergänzende Hilfe
\newenvironment{hilfe}{
	\begin{tcolorbox}[sharp corners, colback=white, colframe=mkgreen, title=\textbf{Hilfe}]
		}
		{
	\end{tcolorbox}
}

% Box für interessante Infos
\newenvironment{info}{
	\begin{tcolorbox}[sharp corners, colback=white, colframe=mkblue, title=\textbf{Info}]
		}
		{
	\end{tcolorbox}
}

% Box für wichtige Hinweise
\newenvironment{achtung}{
	\begin{tcolorbox}[sharp corners, colback=white, colframe=mkyellow, title=\textbf{Achtung}]
		}
		{
	\end{tcolorbox}
}

% Eine nach rechts zusammenfassende Klammer
\newenvironment{rcases} {
	\left.\begin{aligned}
		}
		{
	\end{aligned}\right\rbrace{}
}

% Für array Umgebungen mit displaystyle math
\newenvironment{bigarray}[1] {
	\everymath={\displaystyle}
	\renewcommand{\arraystretch}{2.5}
	\begin{array}{#1}
		}
		{
	\end{array}
}
