% Box für Übungen
\newenvironment{uebung}[1][\unskip]{
	\begin{tcolorbox}[sharp corners, colback=white, colframe=orange, title=\textbf{Übung #1}]
		}
		{
	\end{tcolorbox}
}

% "Box" für Aufgabe
\newenvironment{aufgabe}{
	\paragraph{Aufgabe}
}{}

% "Box" für Lösungsweg
\newenvironment{loesungsweg}{
	\paragraph{Lösungsweg}
}{}

% "Box" für Lösung
\newenvironment{loesung}{
	\paragraph{Lösung}
}{}

% Box für Gesetze, die hervorgehoben werden sollen
\newenvironment{gesetz}{
	\begin{tcolorbox}[sharp corners, colback=white, colframe=mkblue]
		}
		{
	\end{tcolorbox}
}

% Box für Gesetze, die hervorgehoben werden sollen
\newenvironment{hilfe}{
	\begin{tcolorbox}[sharp corners, colback=white, colframe=mkgreen]
		}
		{
	\end{tcolorbox}
}

% Box für interessante Infos
\newenvironment{info}{
	\begin{tcolorbox}[sharp corners, colback=white, colframe=mkblue, title=\textbf{Info}]
		}
		{
	\end{tcolorbox}
}

% Box für wichtige Hinweise
\newenvironment{achtung}{
	\begin{tcolorbox}[sharp corners, colback=white, colframe=mkred, title=\textbf{Achtung}]
		}
		{
	\end{tcolorbox}
}

% Eine nach rechts zusammenfassende Klammer
\newenvironment{rcases} {
	\left.\begin{aligned}
		}
		{
	\end{aligned}\right\rbrace{}
}

% Für array Umgebungen mit displaystyle math
\newenvironment{bigarray}[1] {
	\everymath={\displaystyle}
	\renewcommand{\arraystretch}{2.5}
	\begin{array}{#1}
}
{
	\end{array}
}
