% Geschweifte Klammer nach unten gerichtet
\newcommand{\ubf}{
    \kern-2\tabcolsep\upbracefill\kern-2\tabcolsep{}
}

% Geschweifte Klammer nach oben gerichtet 
\newcommand{\obf}{
    \kern-2\tabcolsep\downbracefill\kern-2\tabcolsep{}
}

% Limes von zu
\newcommand{\limfromto}[2]{
    \lim_{#1 \rightarrow{} #2}
}

% Limes von * nach unendlich
\newcommand{\limtoinfty}[1]{
    \limfromto{#1}{\infty}
}

% Limes von * nach - unendlich
\newcommand{\limtomininfty}[1]{
    \limfromto{#1}{-\infty}
}

% Limes von * nach 0
\newcommand{\limtozero}[1]{
    \limfromto{#1}{0}
}

% Entspricht-Zeichen
\newcommand{\entspricht}{
    \ \widehat{=}\
}

% Hervorhebung
\newcommand{\highlight}[1]{
    \colorbox{orange!50}{\(\displaystyle#1\)}
}

\newcommand{\verteq}{
    \mid\mid{}
}

% Schriftart: Sans-Serif
%\renewcommand{\rmdefault}{\sfdefault}

% Itemize Lists: Aufzählungszeichen
\renewcommand\labelitemi{\textbullet}

% Ableitungs "d"
\newcommand*\diff{\mathop{}\!\mathrm{d}}

% Eulersche Konstante
\newcommand*\e{\mathop{}\!\mathrm{e}}

% Arkuskotangens
\DeclareMathOperator{\arccot}{arccot}

% Logarithmus dualis
\DeclareMathOperator{\ld}{ld}

% Binärer Logarithmus 
\DeclareMathOperator{\lb}{lb}

% Behebt Inputfehler für *.pgf Dateien
\DeclareUnicodeCharacter{2212}{-}

% Unterstrichene Links
\newcommand{\uhref}[2]{\underline{\href{#1}{#2}}}
